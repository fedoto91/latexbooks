% LaTeX source for Introduction to Computer Science and Programming (Java)
% Primary Author:  Seth D. Bergmann

%% Partial class diagram of the Exception classes
%% from java.util


%% \documentclass{article}
%% \usepackage{../../cmds}
%% \usepackage{../../DraTex}
%% \usepackage{../../AlDraTex}
%% \begin{document}

\begin {figure}
\ArrowHeads(1)
\ArrowSpec(H)
\Define \MyEdge(2)
  { \Edge(#2,#1) }
\TreeSpec(\RectNode)() (\MyEdge)
\Draw

\Tree()
  (2,\underline{Throwable}~~~~				//
   0,\underline{Error}~~~~			
	& 2,\underline{Exception}~~ ~~			//
   2,\underline{RunTimeException}~~ \symbol{40}unchecked\symbol{41} ~~	
	& 1,\underline{IOException}~~ \symbol{40}checked\symbol{41} ~~	//
   0,\underline{NullPointerException}~~ \symbol{40}unchecked\symbol{41} ~~	
%% 	& 0,\underline{IndexOutOfBoundsException}~~ \symbol{40}unchecked\symbol{41} ~~	
	& 0,\underline{IllegalArgumentException}~~ \symbol{40}unchecked\symbol{41} ~~	
	& 0,\underline{FileNotFoundException}~~ \symbol{40}checked\symbol{41} ~~	//
  )

%%%% 
%% /symbol{40} is left parenthesis
%% /symbol{41} is right parenthesis
%% Maybe you can find a better way to include parentheses in a tree.



\EndDraw

\caption {Class diagram showing some of the 
Exception classes in the Java class library}


\label {fig:ExceptionClasses}

\end {figure}

%% \end {document}
