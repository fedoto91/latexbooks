% LaTeX source for Introduction to Computer Science and Programming (Java)
% Primary Author:  Seth D. Bergmann

%% Diagram of a frame with nested containers


%% \documentclass{article}
%% \usepackage{../../cmds}
%% \usepackage{../../DraTex}
%% \usepackage{../../AlDraTex}
%% \begin{document}
  
\begin {figure}

\Draw

\MinNodeSize(128,32)

\RectNode(n)(--NORTH
		(Flow layout)--)


\MinNodeSize(94,64)
\Move(17,-48)
\RectNode(c)(--CENTER
		(Label)--)

\MinNodeSize(34,64)
\Move(-64,0)
\RectNode(w)(--WEST~~~~  
		(Grid)--)

\MinNodeSize(38,40)
\Move(2,-52)
\RectNode(sw)(--SOUTH
		WEST--)

\MinNodeSize(52,40)
\Move(45,0)
\RectNode(sc)(--SOUTH
		(Border)
		CENTER--)

\MinNodeSize(38,40)
\Move(45,0)
\RectNode(se)(--SOUTH
		EAST--)

%% Show grid
\PenSize(0.1pt)
\MoveToLoc(w)
\Move(0,32) \Line(0,-64)
\Move(-16,21) \Line(32,0)
\Move(0,22) \Line(-32,0)

\MoveToLoc(n)
\Move(150,-30)
\Node(cp)(--contentPane
	    (Border layout)--)

\ArrowHeads(1)
\CurvedEdgeAt(cp,-1,0,n,1,1)
	     (-180,0.3,0,0.3)



\EndDraw

\caption {A contentPane with BorderLayout, in which containers
have been placed in the north, west, and south regions -- the container
in the south region also uses BorderLayout}

\label {fig:nestedContainers}

\end {figure}

%% \end {document}

