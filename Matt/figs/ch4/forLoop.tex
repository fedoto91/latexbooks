% LaTeX source for Introduction to Computer Science and Programming (Java)
% Primary Author:  Seth D. Bergmann

%% \documentclass{article}

%% \usepackage{../../cmds}
%% \usepackage{../../DraTex}
%% \usepackage{../../AlDraTex}


%%  \begin{document}
%%  Flow diagram for a for loop

\begin {figure}

\Draw

\FcNode(A)
\Move(0,-35)

\MinNodeSize(170,20)
\RectNode(init)(--initilization--)
{ \Move(0,-60)
\MinNodeSize(170,100)
\RectNode(big)(-- --)
}


\Move(0,-25)
\FcNode(I)

\MinNodeSize(20,20)
\Move(0,-35)
\DiamondNode(cond)(--condition--)
\Move(50,10)
\Node(false)(--false--)
\MoveToLoc(cond)
\Move(-15,-25)
\Node(true)(--true--)




\MoveToLoc(I)
\Move(-50,0)
\RectNode(incr)(--increment--)

\MoveToLoc(cond)
\Move(0,-75)
\RectNode(stmt)(--statement--)

\Move(0,-25)
\FcNode(F)

\Move(-30,0)
\FcNode(G)

\Move(-75,0)	 %% use HVEdge to connect G with incr
\FcNode(G)

\MoveToLoc(cond)
\Move(100,0)
\FcNode(B)

\MoveToLoc(F)
\Move(0,-25)
\FcNode(D)	 %% HVEdge, D to B

\Move(0,-25)
\FcNode(E)

\ArrowHeads(1)
\Edge(A,init)
\Edge(D,E)  
\VHEdge(G,incr)
\Edge(cond,stmt)
\Edge(incr,I)
\Edge(init,cond)

\ArrowHeads(0)
\Edge(cond,B)
\VHEdge(B,D)
\Edge(stmt,F)  
\Edge(F,G)  



\EndDraw


\caption {Flow diagram for a \texttt{for} loop structure}

\label {fig:forLoop}

\end {figure}

%% \end {document}

