% LaTeX source for Introduction to Computer Science and Programming (Java)
% Primary Author:  Seth D. Bergmann

%% Draw a diagram depicting method abstraction
%% Interface vs Implementation


%% \documentclass{article}

%% \usepackage{../../cmds}
%% \usepackage{../../DraTex}
%% \usepackage{../../AlDraTex}


%% \begin{document}

\begin {figure}


\Draw

%% \MinNodeSize (20,10)
\ArrowHeads(1)
\ArrowSpec(F,10,4,5)
%% \DecVar \MaxX

%% Display the primitive values in an object
%% Called once for each primitive field
\Define \Var(3)
{	
%% parms:
%%    #1: 0 if this var has no primitive value
%%    #2: name of the var
%%    #3: value in a box
	
   \IF \GtInt (#1,0)  \THEN
	\MarkLoc(left)
 	\Move(0,-27)
	\EntryExit(-1,0,-1,0)
	\Node (tag) (--#2--)
	%% Note the longest field value, to draw the box
	{ 
	\MoveToExit(1,0)
	  \ORectNode (tag) (--#3--)
	\MoveToExit(1,0)  \MarkLoc(right)
	\Move(50,0)  \MarkLoc(rightPlus) 
	\MoveToLL(right,rightPlus)(topL,maxL)
	\MarkGLoc(bottomL)  
	\LSeg\Q(maxL,maxRight)
	\LSeg\R(bottomL,right)
	\IF \GtDec (\R,\Q) \THEN
	    \MoveToLoc(right)
	    \MarkGLoc(maxRight)
	    \FI
	}
	\EntryExit(0,0,0,0)
\MarkGLoc(bottom)
   \FI
}


%% Display the object references in an object
%% Called once for each primitive field
%% Object is a recursive data structure
\Define \Obj(8)
{	
%% parms:
%%    #1: 0 do not display
%%        1 display ordinary ref
%%	  2 display ref in a set
%%        3 display a key in a map
%%        4 display a ref to a set
%%        5 display a ref to a map
%%  	  6 display object only, calling OD directly, no arrow
%%    #2: name of the field
%%    #3: class of the field  (for object types)
%%    #4: table of primitive fields in the referred object
%% 	   or size for a map
%%    #5: table of object fields in the referred object
%%         or table of primitive keys for maps
%%    #6: table of object keys in the referred object
%%    #7: table of primitive values in the referred object
%%    #8: table of object values in the referred object

  % Base case for recursion:  #1 = 0

\IF \GtInt (#1, 0) \THEN		% Recursive case
					% for ordinary field or value in map

   \Move(0,-25)
   \MarkLoc(tmp)
   \EntryExit(-1,0,1.1,0)
   \Node (tag) (--#2--)
   \ORectNode(ref) (-- --)
   \EntryExit(0,0,0,0)
   \IF \EqInt(#1,1) \THEN		% ordinary ref
      \Move(100,\Val\I)
      \OD(#3, #4, #5, 1)  
   \FI
   \IF \EqInt(#1,2) \THEN		% ref in a set
 	\ODset(#3,#4,#5)
   \FI
   \IF \EqInt(#1,3) \THEN		% key in a map
	    \Move(-160,\Val\I)
	    \Move(0,-10)    
	    \OD(#3, #4, #5,3)  
   \FI
   \IF \EqInt(#1,4) \THEN		% ref to a set
	       \OD(#3,#4,#5,4)
   \FI
   \IF \EqInt(#1,5) \THEN		% ref to a map
	          \ODmap(#3,#4,#5,#6,#7,#8,#1)
   \FI
   \MoveToLoc(tmp)
   \I-100;
   \MarkGLoc(bottom)
\FI
   }
 
%% Draw the object in a rectangular box
%% Show primitive values
%% Show references as arrows to objects
\Define \OD(4)
{  
%% parms:
%%    #1:  class of the object 
%%    #2:  table of primitive fields 
%%    #3:  table of object fields 
%%    #4:  See parm #1 for Obj above

   \SaveAll
%%%%  J is depth of recursion
\J+1;
\IF \GtInt (\J,2) \THEN
      \Move(-400,-200)
      \J=0;
    
\FI
\Move (-20,60)   %%% ???  
\Node(obj)(--\underline{#1}--)
%% topLeft and bottomRight are used to draw the rectangle
\MoveToExit(1.02,0)    \MarkGLoc(maxRight)   %% maxRight stores largest primitive
\MoveToExit(-1.2,1.2)  \MarkLoc(topLeft)
\Move (50,-70)
 
\MinNodeSize(30,15)
\PenSize(0.25pt)	%% Lightweight arrow, might cross over other objects

%% Draw the arrow for a reference
\IF \EqInt (#4,3) \THEN 	%% key in a map
     { \CurvedEdgeAt (ref,-0.8,0, obj,1,0.5)(180,0.6,90, 0.6) }
	\ELSE
	\IF \EqInt (#4,6) \THEN		%% calling OD directly 
		{ } 			%% no arrow
        \ELSE
     	{ \CurvedEdgeAt (ref,0.8,0, obj,-1,0.8)(0,0.6,180, 0.6) }
   	\FI
   \FI

\Move(-45,70)
\PenSize(0.75pt)

{\MarkGLoc (maxL) \Move (0,100) \MarkGLoc(topL) }

\Indirect <#2>(0,9) { \Var }		%% Draw primitive fields
\MarkLoc(bottomTemp)
\MoveToLoc(maxRight) \MarkLoc(maxRightLocal)
\MoveToLoc(bottomTemp)

\Indirect<#3>(0,8) { \Obj   }		%% Draw object fields

%% Set maxRightLocal in case there were no primitive fields
\Move(100,0) \MarkLoc(maxRightTemp)
\MoveToRightmost(maxRightLocal,maxRightTemp)
\MarkLoc(maxRightLocal)
{
%% Find bottom right corner of rectangle
\MoveToLoc(bottom)
\Move(100,0) \MarkLoc(bottomR)
\MoveToLoc(maxRightLocal) \Move(0,-100) \MarkLoc(rightB)
\MoveToLL(bottom,bottomR)(maxRightLocal,rightB)
\Move(10,-20) 
\MarkLoc(bottomRight)
\CSeg \DrawRect(bottomRight,topLeft)
}
   
   \RecallAll
\J-1;
}

%% Draw a ref to a Set
\Define \ObjSet(6)
{	
%% parms:
%%    #1: 0 iff this field has no Object ref
%%    #2: name of the field
%%    #3: class of the field  (for object types)
%%    #4: table of primitive fields in the referred object - wrappers or Strings
%%    #5: table of object fields in the referred object
%%    #6: size of the set 

  % Base case:  #1 = 0

\IF \GtInt (#1, 0) \THEN		% Recursive case
	\MarkLoc(tmp)
   	\Node (tag) (--#2--)
	\Move (50,0)
	\ORectNode(ref) (-- --)
	\Move(50,\Val\I)
	\Move (35,20)
	\ODset(#3, #4, #5,#6)  
	\MoveToLoc(tmp)
	\I-135;
   \FI
}

%% Draw a set with the refs at 'random' places in the box
\Define \ODset(4)    
{  
%% parms:
%%    #1:  class of the object
%%    #2:  table of primitive fields 
%%    #3:  table of object fields 
%%    #4:  size of the set        
   \SaveAll
  
\MinNodeSize (90,110)
\RectNode(obj)(-- #1~~~~~~~~~~~~~~~~~~~~~~~--)
\MinNodeSize (20,10)

{ \PenSize(0.25pt)	%% Lightweight arrow, might cross over other objects
  %% Draw arrow to this Set object
  \CurvedEdgeAt (ref,0.8,0, obj,-1,0.8)(10,0.6,180, 0.6) }
  \PenSize(0.75pt)
   \IntVar \X      
   \IntVar \Y

   \Move(-50,70)
   { \Line(110,0) }
   \Move (15,-15)
   \Node(tag)(--size--)
   \MoveToExit(1,0)
   \Move(25,0)
   \ORectNode(tag)(--#4--)
   \Move(-90,0)
   \MarkLoc(start)

%% Place the fields at seemingly random locations
   \Indirect <#2>(0,9) { 		%% Draw the primitive fields
 	\X+34;
 	\K=\X;
	\K/90;
	\K*90;
 	\X-\K;		%% x + 34 mod 90
	\Y+73;
	\K=\Y;
	\K/110;
	\K*110;
	\Y-\K;		%% y + 73 mod 110
	\Y+20;		%% Allow for size at top
 		\MoveToLoc(start)
		\Move (\Val\X,-\Val\Y)
 		\Var
}

%% Place the fields at seemingly random locations
   \Indirect<#3>(0,9) { 		%% Draw the object fields
 	\X+34;
 	\K=\X;
 	\K/90;
 	\K*90;
 	\X-\K;		%% x + 72 mod 90
	\Y+73;
	\K=\Y;
	\K/110;
	\K*110;
	\Y-\K;		%% y + 73 mod 110
	\Y+20;		%% Allow for size at top
 	\MoveToLoc(start)
	\Move (\Val\X,-\Val\Y)
		 \Obj   
	}
   \RecallAll
}

%% Draw a map object in a rectangle
%% Value refs point to the right
%% Key refs point to the left
%% Note largest value field to draw the rectangle
\Define \ODmap(7)
{  
%% parms:
%%    #1:  class of the object 
%%    #2:  size
%%    #3:  table of primitive keys 
%%    #4:  table of object keys 
%%    #5:  table of primitive values 
%%    #6:  table of object values 
%%    #7:  see parm #1 for Obj above
   \SaveAll
  
\Move(-20,-60) 
\Node(obj)(--\underline{#1}--)
\MoveToExit(1.02,0) 	\MarkLoc (maxRightLocal)
			\MarkLoc (maxRight)
\MoveToExit(-1.2,1.2)\MarkLoc (topLeft)

\MinNodeSize(30,15)
\PenSize(0.25pt)		%% Use Lightweight pen, may cross other objects

\IF \EqInt(#7,3) \THEN
     \CurvedEdgeAt (ref,-0.8,0, obj,1,0.5)(180,0.6,90, 0.6) 
     \ELSE
	\IF \EqInt(#7,6) \THEN		%% calling ODmap directly
	    {  }
	\ELSE
	  \CurvedEdgeAt (ref,0.8,0, obj,-1,0.8)(10,0.6,180, 0.6) 
        \FI
\FI
	
\PenSize(0.75pt)
\Move(20,-35)
   
   %% Show size and headings for keys and values
   { \Node(tag)(--size--) \Move(40,0) \ORectNode(tag)(--#2--) }
   \Move (10,-20)
   { \Node(tag)(--\underline{keys}--) 
     \Move(75,0) \Node(tag)(--\underline {values}--) 
   }
   \Move (-50,0)
   \MarkLoc(keys)

   {\MarkGLoc (maxL) \Move (0,100) \MarkGLoc(topL) }   %% ??

  %% keys 
   \Indirect <#3>(0,9) { \Var  }	%% Draw the primitive keys

   \Indirect<#4>(0,8) { \Obj   }	%% Draw the object keys
   \MarkGLoc(bottom)
\MoveToLoc(keys)
\Move(83, -1)
{
  %% values 
   \Indirect <#5>(0,9) { \Var  }	%% Draw the primitive values
   \MarkLoc(bottomTemp)
   \MoveToLoc(bottomTemp)

   \Indirect<#6>(0,8) { \Obj   }	%% Draw the object values

%% Keys may not be primitive, affects size of box
\Move(100,0)	\MarkLoc(maxRightTemp)
\MoveToRightmost(maxRightLocal,maxRightTemp)
\MarkLoc(maxRightLocal)
}
{
%% Find bottom right corner of rectangle
\MoveToLoc(bottom)
\Move(100,0)	\MarkLoc(bottomR)
\MoveToLoc(maxRightLocal)	\Move(0,-100)	\MarkLoc(rightB)
\MoveToLL(bottom,bottomR)(maxRightLocal,rightB)
\Move(10,-20)
\MarkLoc(bottomRight)
\CSeg \DrawRect(bottomRight,topLeft)
}
   \RecallAll
}

%% Move the pen to the loc which is furthest right
\Define \MoveToRightmost(2)
{
%% #1:  A loc
%% #2:  A loc
\MoveTo(-200,100)	\MarkLoc(upLeft)
\MoveTo(-200,-100)	\MarkLoc(downLeft)
\MoveToLoc(#1)		\Move(100,0)	\MarkLoc(right)
\MoveToLL(upLeft,downLeft)(#1,right)	\MarkLoc(leftA)
\MoveToLoc(#2)		\Move(100,0)	\MarkLoc(right)
\MoveToLL(upLeft,downLeft)(#2,right)	\MarkLoc(leftB)

	\LSeg\Q(leftA,#1)
	\LSeg\R(leftB,#2)
	\MoveToLoc(#1)
	\IF \GtDec (\R,\Q) \THEN
	    \MoveToLoc(#2)
	    \FI
}

\MinNodeSize (20,10)
\ArrowHeads(1)
\ArrowSpec(F,10,4,5)
%% \DecVar \MaxX

%% Display the primitive values in an object
%% Called once for each primitive field
\Define \Var(3)
{	
%% parms:
%%    #1: 0 if this var has no primitive value
%%    #2: name of the var
%%    #3: value in a box
	
   \IF \GtInt (#1,0)  \THEN
	\MarkLoc(left)
 	\Move(0,-27)
	\EntryExit(-1,0,-1,0)
	\Node (tag) (--#2--)
	%% Note the longest field value, to draw the box
	{ 
	\MoveToExit(1,0)
	  \ORectNode (tag) (--#3--)
	\MoveToExit(1,0)  \MarkLoc(right)
	\Move(50,0)  \MarkLoc(rightPlus) 
	\MoveToLL(right,rightPlus)(topL,maxL)
	\MarkGLoc(bottomL)  
	\LSeg\Q(maxL,maxRight)
	\LSeg\R(bottomL,right)
	\IF \GtDec (\R,\Q) \THEN
	    \MoveToLoc(right)
	    \MarkGLoc(maxRight)
	    \FI
	}
	\EntryExit(0,0,0,0)
\MarkGLoc(bottom)
   \FI
}


%% Display the object references in an object
%% Called once for each primitive field
%% Object is a recursive data structure
\Define \Obj(8)
{	
%% parms:
%%    #1: 0 do not display
%%        1 display ordinary ref
%%	  2 display ref in a set
%%        3 display a key in a map
%%        4 display a ref to a set
%%        5 display a ref to a map
%%  	  6 display object only, calling OD directly, no arrow
%%    #2: name of the field
%%    #3: class of the field  (for object types)
%%    #4: table of primitive fields in the referred object
%% 	   or size for a map
%%    #5: table of object fields in the referred object
%%         or table of primitive keys for maps
%%    #6: table of object keys in the referred object
%%    #7: table of primitive values in the referred object
%%    #8: table of object values in the referred object

  % Base case for recursion:  #1 = 0

\IF \GtInt (#1, 0) \THEN		% Recursive case
					% for ordinary field or value in map

   \Move(0,-25)
   \MarkLoc(tmp)
   \EntryExit(-1,0,1.1,0)
   \Node (tag) (--#2--)
   \ORectNode(ref) (-- --)
   \EntryExit(0,0,0,0)
   \IF \EqInt(#1,1) \THEN		% ordinary ref
      \Move(100,\Val\I)
      \OD(#3, #4, #5, 1)  
   \FI
   \IF \EqInt(#1,2) \THEN		% ref in a set
 	\ODset(#3,#4,#5)
   \FI
   \IF \EqInt(#1,3) \THEN		% key in a map
	    \Move(-160,\Val\I)
	    \Move(0,-10)    
	    \OD(#3, #4, #5,3)  
   \FI
   \IF \EqInt(#1,4) \THEN		% ref to a set
	       \OD(#3,#4,#5,4)
   \FI
   \IF \EqInt(#1,5) \THEN		% ref to a map
	          \ODmap(#3,#4,#5,#6,#7,#8,#1)
   \FI
   \MoveToLoc(tmp)
   \I-100;
   \MarkGLoc(bottom)
\FI
   }
 
%% Draw the object in a rectangular box
%% Show primitive values
%% Show references as arrows to objects
\Define \OD(4)
{  
%% parms:
%%    #1:  class of the object 
%%    #2:  table of primitive fields 
%%    #3:  table of object fields 
%%    #4:  See parm #1 for Obj above

   \SaveAll
%%%%  J is depth of recursion
\J+1;
\IF \GtInt (\J,2) \THEN
      \Move(-400,-200)
      \J=0;
    
\FI
\Move (-20,60)   %%% ???  
\Node(obj)(--\underline{#1}--)
%% topLeft and bottomRight are used to draw the rectangle
\MoveToExit(1.02,0)    \MarkGLoc(maxRight)   %% maxRight stores largest primitive
\MoveToExit(-1.2,1.2)  \MarkLoc(topLeft)
\Move (50,-70)
 
\MinNodeSize(30,15)
\PenSize(0.25pt)	%% Lightweight arrow, might cross over other objects

%% Draw the arrow for a reference
\IF \EqInt (#4,3) \THEN 	%% key in a map
     { \CurvedEdgeAt (ref,-0.8,0, obj,1,0.5)(180,0.6,90, 0.6) }
	\ELSE
	\IF \EqInt (#4,6) \THEN		%% calling OD directly 
		{ } 			%% no arrow
        \ELSE
     	{ \CurvedEdgeAt (ref,0.8,0, obj,-1,0.8)(0,0.6,180, 0.6) }
   	\FI
   \FI

\Move(-45,70)
\PenSize(0.75pt)

{\MarkGLoc (maxL) \Move (0,100) \MarkGLoc(topL) }

\Indirect <#2>(0,9) { \Var }		%% Draw primitive fields
\MarkLoc(bottomTemp)
\MoveToLoc(maxRight) \MarkLoc(maxRightLocal)
\MoveToLoc(bottomTemp)

\Indirect<#3>(0,8) { \Obj   }		%% Draw object fields

%% Set maxRightLocal in case there were no primitive fields
\Move(100,0) \MarkLoc(maxRightTemp)
\MoveToRightmost(maxRightLocal,maxRightTemp)
\MarkLoc(maxRightLocal)
{
%% Find bottom right corner of rectangle
\MoveToLoc(bottom)
\Move(100,0) \MarkLoc(bottomR)
\MoveToLoc(maxRightLocal) \Move(0,-100) \MarkLoc(rightB)
\MoveToLL(bottom,bottomR)(maxRightLocal,rightB)
\Move(10,-20) 
\MarkLoc(bottomRight)
\CSeg \DrawRect(bottomRight,topLeft)
}
   
   \RecallAll
\J-1;
}

%% Draw a ref to a Set
\Define \ObjSet(6)
{	
%% parms:
%%    #1: 0 iff this field has no Object ref
%%    #2: name of the field
%%    #3: class of the field  (for object types)
%%    #4: table of primitive fields in the referred object - wrappers or Strings
%%    #5: table of object fields in the referred object
%%    #6: size of the set 

  % Base case:  #1 = 0

\IF \GtInt (#1, 0) \THEN		% Recursive case
	\MarkLoc(tmp)
   	\Node (tag) (--#2--)
	\Move (50,0)
	\ORectNode(ref) (-- --)
	\Move(50,\Val\I)
	\Move (35,20)
	\ODset(#3, #4, #5,#6)  
	\MoveToLoc(tmp)
	\I-135;
   \FI
}

%% Draw a set with the refs at 'random' places in the box
\Define \ODset(4)    
{  
%% parms:
%%    #1:  class of the object
%%    #2:  table of primitive fields 
%%    #3:  table of object fields 
%%    #4:  size of the set        
   \SaveAll
  
\MinNodeSize (90,110)
\RectNode(obj)(-- #1~~~~~~~~~~~~~~~~~~~~~~~--)
\MinNodeSize (20,10)

{ \PenSize(0.25pt)	%% Lightweight arrow, might cross over other objects
  %% Draw arrow to this Set object
  \CurvedEdgeAt (ref,0.8,0, obj,-1,0.8)(10,0.6,180, 0.6) }
  \PenSize(0.75pt)
   \IntVar \X      
   \IntVar \Y

   \Move(-50,70)
   { \Line(110,0) }
   \Move (15,-15)
   \Node(tag)(--size--)
   \MoveToExit(1,0)
   \Move(25,0)
   \ORectNode(tag)(--#4--)
   \Move(-90,0)
   \MarkLoc(start)

%% Place the fields at seemingly random locations
   \Indirect <#2>(0,9) { 		%% Draw the primitive fields
 	\X+34;
 	\K=\X;
	\K/90;
	\K*90;
 	\X-\K;		%% x + 34 mod 90
	\Y+73;
	\K=\Y;
	\K/110;
	\K*110;
	\Y-\K;		%% y + 73 mod 110
	\Y+20;		%% Allow for size at top
 		\MoveToLoc(start)
		\Move (\Val\X,-\Val\Y)
 		\Var
}

%% Place the fields at seemingly random locations
   \Indirect<#3>(0,9) { 		%% Draw the object fields
 	\X+34;
 	\K=\X;
 	\K/90;
 	\K*90;
 	\X-\K;		%% x + 72 mod 90
	\Y+73;
	\K=\Y;
	\K/110;
	\K*110;
	\Y-\K;		%% y + 73 mod 110
	\Y+20;		%% Allow for size at top
 	\MoveToLoc(start)
	\Move (\Val\X,-\Val\Y)
		 \Obj   
	}
   \RecallAll
}

%% Draw a map object in a rectangle
%% Value refs point to the right
%% Key refs point to the left
%% Note largest value field to draw the rectangle
\Define \ODmap(7)
{  
%% parms:
%%    #1:  class of the object 
%%    #2:  size
%%    #3:  table of primitive keys 
%%    #4:  table of object keys 
%%    #5:  table of primitive values 
%%    #6:  table of object values 
%%    #7:  see parm #1 for Obj above
   \SaveAll
  
\Move(-20,-60) 
\Node(obj)(--\underline{#1}--)
\MoveToExit(1.02,0) 	\MarkLoc (maxRightLocal)
			\MarkLoc (maxRight)
\MoveToExit(-1.2,1.2)\MarkLoc (topLeft)

\MinNodeSize(30,15)
\PenSize(0.25pt)		%% Use Lightweight pen, may cross other objects

\IF \EqInt(#7,3) \THEN
     \CurvedEdgeAt (ref,-0.8,0, obj,1,0.5)(180,0.6,90, 0.6) 
     \ELSE
	\IF \EqInt(#7,6) \THEN		%% calling ODmap directly
	    {  }
	\ELSE
	  \CurvedEdgeAt (ref,0.8,0, obj,-1,0.8)(10,0.6,180, 0.6) 
        \FI
\FI
	
\PenSize(0.75pt)
\Move(20,-35)
   
   %% Show size and headings for keys and values
   { \Node(tag)(--size--) \Move(40,0) \ORectNode(tag)(--#2--) }
   \Move (10,-20)
   { \Node(tag)(--\underline{keys}--) 
     \Move(75,0) \Node(tag)(--\underline {values}--) 
   }
   \Move (-50,0)
   \MarkLoc(keys)

   {\MarkGLoc (maxL) \Move (0,100) \MarkGLoc(topL) }   %% ??

  %% keys 
   \Indirect <#3>(0,9) { \Var  }	%% Draw the primitive keys

   \Indirect<#4>(0,8) { \Obj   }	%% Draw the object keys
   \MarkGLoc(bottom)
\MoveToLoc(keys)
\Move(83, -1)
{
  %% values 
   \Indirect <#5>(0,9) { \Var  }	%% Draw the primitive values
   \MarkLoc(bottomTemp)
   \MoveToLoc(bottomTemp)

   \Indirect<#6>(0,8) { \Obj   }	%% Draw the object values

%% Keys may not be primitive, affects size of box
\Move(100,0)	\MarkLoc(maxRightTemp)
\MoveToRightmost(maxRightLocal,maxRightTemp)
\MarkLoc(maxRightLocal)
}
{
%% Find bottom right corner of rectangle
\MoveToLoc(bottom)
\Move(100,0)	\MarkLoc(bottomR)
\MoveToLoc(maxRightLocal)	\Move(0,-100)	\MarkLoc(rightB)
\MoveToLL(bottom,bottomR)(maxRightLocal,rightB)
\Move(10,-20)
\MarkLoc(bottomRight)
\CSeg \DrawRect(bottomRight,topLeft)
}
   \RecallAll
}

%% Move the pen to the loc which is furthest right
\Define \MoveToRightmost(2)
{
%% #1:  A loc
%% #2:  A loc
\MoveTo(-200,100)	\MarkLoc(upLeft)
\MoveTo(-200,-100)	\MarkLoc(downLeft)
\MoveToLoc(#1)		\Move(100,0)	\MarkLoc(right)
\MoveToLL(upLeft,downLeft)(#1,right)	\MarkLoc(leftA)
\MoveToLoc(#2)		\Move(100,0)	\MarkLoc(right)
\MoveToLL(upLeft,downLeft)(#2,right)	\MarkLoc(leftB)

	\LSeg\Q(leftA,#1)
	\LSeg\R(leftB,#2)
	\MoveToLoc(#1)
	\IF \GtDec (\R,\Q) \THEN
	    \MoveToLoc(#2)
	    \FI
}

\MinNodeSize(220,30)
\RectNode(tag)(--Useful tool--)
\Move(-55,-65)

\MinNodeSize(110,100)
\RectNode(tag)(--\underline{Interface}
		(Exposed)~~
		What you need to
		know to use this
		tool
		(API)--)
{ \MoveToExit(-1,0.8) \MarkGLoc(int)      }
{ \MoveToExit(-2,0)  \MarkGLoc(interface)  }

\MoveToExit (2,0)
\RectNode(tag)(--\underline{Implementation}
		(Hidden from view)~~
		Details:
		How this tool is
		constructed~~--)

\MoveToLoc(interface)
\Move(0,50)
\DrawCircle(10)
{ \Move(6,3) \PaintCircle(1) \Move (5,0) \MarkGLoc(eye)  	% eye
  \Move(-4,0) \DrawOvalArc (9,9)(-135,-90)			% mouth
}
\Move (0,-10)
\Line(0,-50)
{ \Move(0,25) \Line(20,20) }
{ \Move(0,25) \Line(-20,20) }
{ \Line(20,-20) }
{ \Line(-20,-20) }

%  line of sight
\MoveToLoc(eye) \FcNode(eyeNode)   
\MoveToLoc(int) \FcNode(intNode)   
\EdgeSpec(D)						% dashed edge
\Edge (intNode,eyeNode)



\EndDraw

\caption {Abstraction: 
There is no need to know the implementation 
details of the tools being used.}


\label {fig:methAbstr1}

\end {figure}

%% \end {document}

