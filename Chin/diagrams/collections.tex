\documentclass{article}
\usepackage{expl3, xparse}% http://ctan.org/pkg/xparse
\usepackage{etoolbox}% http://ctan.org/pkg/etoolbox
\usepackage{ifthen}
\usepackage{adjustbox}
\usepackage{tikz}
\usepackage{cmds}
\usepackage{DraTex}
\usepackage{AlDraTex}
\usepackage{pgffor}
\usepackage{graphicx}

\usetikzlibrary{matrix,scopes}
\usetikzlibrary{graphs}
\usetikzlibrary{calc,shapes,shadows,shapes.multipart,chains,matrix,positioning,arrows.meta,arrows}
\usetikzlibrary{calc,shapes.multipart,chains,arrows,positioning}

%%%%%%%%%%%%%%%%%%%%%%%%%%%%%%%%%%%%%%%%%%%%%%%%%%%%%%%%%%%%%%%%%%%%%%%%%%%%%%%%%%%%%%%%%
%% Tikz stylesheet
%%%%%%%%%%%%%%%%%%%%%%%%
\tikzset{every node/.style={inner sep=1pt},
  pre/.style={draw,fill=yellow}}
  
%%%%%%%%%%%%%%%%%%%%%%%%%%%%%%%%%%%%%%%%%%%%%%%%%%%%%%%%%%%%%%%%%%%%%% 
%% Helpers    %%%
%%%%%%%%%%%%%%%%%

\ExplSyntaxOn

\seq_new:N \g_parsed_seq %%global tmp parsed sequence


%%%%%%%%%%%%%%%%%%%%%%%%%%%%
%%%List Processors %%%%%%%%%
%%%%%%%%%%%%%%%%%%%%%%%%%%%%

\NewDocumentCommand{\newlist}{ m }
 {
  \seq_new:c { #1_seq }
 }
 
\NewDocumentCommand{\addtolist}{ O{,} m m }
 {
 \seq_gclear:N \g_parsed_seq %%clear tmp
 \seq_gset_split:Nnn \g_parsed_seq {#1} {#3} %%split list by delimiter, store in tmp seq
 
 %%map function: add each element in tmp in given seq
 \seq_map_inline:Nn \g_parsed_seq
  {
   \seq_gput_right:cn { #2_seq } { ##1 }
  }
 
 }
 
\NewDocumentCommand{\getsize}{m}
{
  \seq_count:c {#1_seq} %%value of list length 
}

\NewDocumentCommand{\getindex}{ m m }
 {
  \seq_item:cn { #1_seq } { #2 }
 }
\NewDocumentCommand{\setindex}{ m m m }
 {
  \cs_set:Npx #3 { \seq_item:cn { #1_seq } { #2 } }
 }
\NewDocumentCommand{\clearlist}{ m }
 {
  \seq_gclear:c { #1_seq }
 }
\NewDocumentCommand{\printlist}{O{~} m}{
\seq_use:cn {#2_seq}{#1}
}
\NewDocumentCommand{\addby}{m m}{%
 \fp_eval:n {\seq_count:c {#1_seq} + (#2)}
}%

\NewDocumentCommand{\ceildivby}{m m}{
\fp_eval:n { ceil((#1) / (#2)) }
}
\ExplSyntaxOff


%%%%%%%%%%%%%%%%%%%%%%%%%%%%%%%%%%%%%%
%%Global Setters
%%%%%%%%%%%%%%%%%%%%%%%%%%%%%%%%%%%%%%%%%%%%%   
 \newlist{link}
 \newlist{test}
 \newcommand{\alignlist}{}
 \newcommand{\alignref}{below}
%%%%%%%%%%%%%%%%%%%%%%%%%%%%%%%%%%%%%%%%%%%%%%%%%%%%%%%%%%%%%%%%%%%%%%%%%%%%%%%%%%%%%%%%%%%%%%%%%
%%% left-right matrix                        %%
%%%                                          %%
%%%%%%%%%%%%%%%%%%%%%%%%%%%%%%%%%%%%%%%%%%%%%%%
%%%%%%%%%%%%%%%%%%%%%%%%%%%%%%%%%%%%%%%%%%%%%%%
\begin{document}

\NewDocumentCommand{\lrmatrix}{ m m }{
\begin{tikzpicture}
  
  \clearlist{test}
  \addtolist{test}{#2}

\let\mymatrix\empty
\newcount\cnt
\cnt=0
\foreach \i in {1,...,\getsize{test}}{
    \begingroup\edef\x{\endgroup
      \noexpand\gappto\noexpand\mymatrix{\getindex{test}{\i} \&}}\x
      \global\advance\cnt by 1
      
    \ifboolexpr{ test {\ifnumcomp{\the\cnt}{=}{#1}} or test {\ifnumcomp{\i}{=}{\getsize{test}}} }
    { \gappto\mymatrix{\\} 
      \global\cnt=0
    }%else
    {}%
  }
   
  \matrix (a) [ampersand replacement=\&,matrix of nodes, row sep=1cm,column sep=1cm]{
    \mymatrix
  };
\end{tikzpicture}
}
%%%%%%%%%%%%%%%%%%%%%%%%%%%%%%%%%%%%%%%%%%%%%%%%%%%%%%%%%%%%%%%%%%%%%%%%%%%%%%%%%%%%%%%%%
%%%vertical matrix by # of rows
%%%
%%%%%%%%%%%%%%%%%%%%%%%%%%%%%%%%%%%%%%%%%%%%%%%%%%%%%%%%%%%%%%%%%%%%%%%%%%%%%%%%%%%%%%%%%
%%%%%%%%%%%%%%%%%%%%%%%%%%%%%%%%%%%%%%%%%%%%
\NewDocumentCommand{\vmatrix}{ m m }{
\begin{tikzpicture}
  
  \clearlist{test}
  \addtolist{test}{#2}

\let\mymatrix\empty

\newcount\from
\global\from=1
\newcount\to
\global\to=#1

\def\xval{\ceildivby{\getsize{test}}{#1}}

\foreach \i in {1,...,\xval}
{
  \foreach \j in {\the\to,...,\the\from}{

    \begingroup\edef\x{\endgroup
      \noexpand\gappto\noexpand\mymatrix{ \getindex{test}{\j} \&} }\x
  }
  \gappto\mymatrix{\\} 
      %%next set
      \global\advance\from by #1
      \global\advance\to by #1
}
%%
\node (aa)[rotate=90] {
\tikz
  \matrix (a) [ampersand replacement= \&,matrix of nodes, row sep=0.5cm,column sep=1cm, nodes={rotate=270}]{
    \mymatrix
  };
  };
\end{tikzpicture}
}
%%%%%%%%%%%%%%%%%%%%%%%%%%%%%%%%%%%%%%%%%%%%%%%%%%%%%%%%%%%%%%%%%%%%%%%%%%%%%%%%%%%%%%%%%
%%%Left-right snake matrix by defined number of columns
%%%
%%%%%%%%%%%%%%%%%%%%%%%%%%%%%%%%%%%%%%%%%%%%%%%%%%%%%%%%%%%%%%%%%%%%%%%%%%%%%%%%%%%%%%%%%
%%%%%%%%%%%%%%%%%%%%%%%%%%%%%%%%%%%%%%%%%%%%
\NewDocumentCommand{\lrsmatrix}{ m m }{
\begin{tikzpicture}
  
  \clearlist{test}
  \addtolist{test}{#2}

\let\mymatrix\empty

\newcount\tmp
\global\tmp=4
\newcount\from
\global\from=1
\newcount\to
\global\to=#1

\def\xval{\ceildivby{\getsize{test}}{#1}}

\foreach \i in {1,...,\xval}
{
  \foreach \j in {\the\from,...,\the\to}{
    \begingroup\edef\x{\endgroup
      \noexpand\gappto\noexpand\mymatrix{ \getindex{test}{\j} \&} }\x
  }
  \gappto\mymatrix{\\} 
      %%reverse
      \global\advance\from by #1
      \global\tmp\from
      \global\advance\to by #1
      \global\from\to
      \global\to\tmp
}

  \matrix (a) [ampersand replacement= \&,matrix of nodes, row sep=0.5cm,column sep=1cm]{
    \mymatrix
  };
\end{tikzpicture}
}
%%%%%%%%%%%%%%%%%%%%%%%%%%%%%%%%%%%%%%%%%%%%%%%%%%%%%%%%%%%%%%%%%%%%%%%%%%%%%%%%%%%%%%%%%
%%%Right-left snake matrix by defined number of columns
%%%
%%%%%%%%%%%%%%%%%%%%%%%%%%%%%%%%%%%%%%%%%%%%%%%%%%%%%%%%%%%%%%%%%%%%%%%%%%%%%%%%%%%%%%%%%
%%%%%%%%%%%%%%%%%%%%%%%%%%%%%%%%%%%%%%%%%%%%
\NewDocumentCommand{\rlsmatrix}{ m m }{
\begin{tikzpicture}
  
  \clearlist{test}
  \addtolist{test}{#2}

\let\mymatrix\empty

\newcount\tmp
\global\tmp=4
\newcount\from
\global\from=1
\newcount\to
\global\to=#1

\def\xval{\ceildivby{\getsize{test}}{#1}}

\foreach \i in {1,...,\xval}
{
  \foreach \j in {\the\to,...,\the\from}{

    \begingroup\edef\x{\endgroup
      \noexpand\gappto\noexpand\mymatrix{ \getindex{test}{\j} \&} }\x
  }
  \gappto\mymatrix{\\} 
      %%reverse
      \global\advance\from by #1
      \global\tmp\from
      \global\advance\to by #1
      \global\from\to
      \global\to\tmp
}
%%
  \matrix (a) [ampersand replacement= \&,matrix of nodes, row sep=0.5cm,column sep=1cm, nodes={}]{
    \mymatrix
  };
\end{tikzpicture}
}
%%%%%%%%%%%%%%%%%%%%%%%%%%%%%%%%%%%%%%%%%%%%%%%%%%%%%%%%%%%%%%%%%%%%%%%%%%%%%%%%%%%%%%%%%
%%%Vertical snake matrix
%%%
%%%%%%%%%%%%%%%%%%%%%%%%%%%%%%%%%%%%%%%%%%%%%%%%%%%%%%%%%%%%%%%%%%%%%%%%%%%%%%%%%%%%%%%%%
%%%%%%%%%%%%%%%%%%%%%%%%%%%%%%%%%%%%%%%%%%%%
\NewDocumentCommand{\vsmatrix}{ m m }{
\begin{tikzpicture}
  
  \clearlist{test}
  
  \addtolist{test}{#2}

\let\mymatrix\empty

\newcount\tmp
\global\tmp=4
\newcount\from
\global\from=1
\newcount\to
\global\to=#1

\def\xval{\ceildivby{\getsize{test}}{#1}}

\foreach \i in {1,...,\xval}
{
  \foreach \j in {\the\to,...,\the\from}{

    \begingroup\edef\x{\endgroup
      \noexpand\gappto\noexpand\mymatrix{ \getindex{test}{\j} \&} }\x
  }
  \gappto\mymatrix{\\} 
      %%reverse
      \global\advance\from by #1
      \global\tmp\from
      \global\advance\to by #1
      \global\from\to
      \global\to\tmp
}

 \node (aa) [rotate=90] {
  \tikz 
  \matrix (a) [ampersand replacement= \&,matrix of nodes, row sep=0.5cm,column sep=1cm, nodes={rotate=270}]{
    \mymatrix
  };
  };
\end{tikzpicture}
}
%%%%%%%%%%%%%%%%%%%%%%%%%%%%%%%%%%%%%%%%%%%%%%%%%%%%%%%%%%%%%%%%
%%%%%%%CS doubly linklist%%%%%%%%%%%%%%%%%%%%%%%%%%%%%%%%%%%%%%%
%%%%param(1):set of values             
%%%%format:\csdlinklist{1,2,3,4}; 
%%%%%%%%%%%%%%%%%%%%%%%%%%%%%%%%%%%%%%%%%%%%%%%%%%%%%%%%%%%%%%%%
\NewDocumentCommand{\csdlinklist}{O{} m}{
\begin{tikzpicture}[every node/.style={align=center},linkn/.style={
            very thick, rectangle split, 
            rectangle split parts=3, draw, 
            rectangle split horizontal, minimum size=18pt,
            inner sep=5pt, text=black,rounded corners,
            rectangle split part fill={blue!20, red!20, blue!20}
            },
        ref/.style={very thick, rectangle split, 
            rectangle split parts=2, draw, 
            , minimum size=18pt,
            inner sep=5pt, text=black,rounded corners,
            rectangle split part fill={blue!20, red!20}}
            ,remember picture, *->, auto, start chain
      ]

	\clearlist{link}           
	
    \addtolist{link}{#2}
    
    \node[linkn, on chain,] (0) {/ \nodepart{second}Head \\ Size = \getsize{link}}; 
    
	%%place and draw of link node and respective ref value
	\foreach \x in {1,...,\getsize{link}}{	 
    \node[linkn,on chain] (\x) {\nodepart{second} Idx(\x)};
    \node[ref, below= of \x] (ref\x) {Ref(\x)-Value \nodepart{second}\getindex{link}{\x}};
    \path ($(\x.center)+(0,-0.12)$) edge (ref\x);
    }
    
    \node[linkn, on chain,] ({\addby{link}{1}}) {\nodepart{second}Tail \nodepart{three} /}; 

  
    \foreach \v [count=\i from 2, count=\j from 0] in {1,...,\getsize{link}}{

     \path ($(\v.three)+(0,0.08)$) edge[bend left] ($(\i.one)+(0,0.08)$);
    \path ($(\v.one)+(0,-0.08)$) edge[bend left] ($(\j.three)+(0,-0.08)$);
    }
    
    \end{tikzpicture}   
}
%%%%%%%%%%%%%%%%%%%%%%%%%%%%%%%%%%%%%%%%%%%%%%%%%%%%%%%%%%%%%%%%
%%%%%%%CS loop doubly linklist%%%%%%%%%%%%%%%%%%%%%%%%%%%%%%%%%%%%%%%
%%%%param(1):set of values             
%%%%format:\csdlinklist{1,2,3,4}; 
%%%%%%%%%%%%%%%%%%%%%%%%%%%%%%%%%%%%%%%%%%%%%%%%%%%%%%%%%%%%%%%%
\NewDocumentCommand{\csloopdlinklist}{O{} m}{
\begin{tikzpicture}[every node/.style={align=center},linkn/.style={
            very thick, rectangle split, 
            rectangle split parts=3, draw, 
            rectangle split horizontal, minimum size=18pt,
            inner sep=5pt, text=black,rounded corners,
            rectangle split part fill={blue!20, red!20, blue!20}
            },
        ref/.style={very thick, rectangle split, 
            rectangle split parts=2, draw, 
            , minimum size=18pt,
            inner sep=5pt, text=black,rounded corners,
            rectangle split part fill={blue!20, red!20}}
            ,remember picture, *->, auto, start chain
      ]

	\clearlist{link}           
	
    \addtolist{link}{#2}
    
    \node[linkn, on chain,] (0) {/ \nodepart{second}Head \\ Size = \getsize{link}}; 
    
	%%place and draw of link node and respective ref value
	\foreach \x in {1,...,\getsize{link}}{	 
    \node[linkn,on chain] (\x) {\nodepart{second} Idx(\x)};
    \node[ref, below= of \x] (ref\x) {Ref(\x)-Value \nodepart{second}\getindex{link}{\x}};
    \path ($(\x.center)+(0,-0.12)$) edge (ref\x);
    }
    
    \node[linkn, on chain,] ({\addby{link}{1}}) {\nodepart{second}Tail \nodepart{three} /}; 

    
    \foreach \v [count=\i from 2, count=\j from 0] in {1,...,\getsize{link}}{

     \path ($(\v.three)+(0,0.08)$) edge[bend left] ($(\i.one)+(0,0.08)$);
    \path ($(\v.one)+(0,-0.08)$) edge[bend left] ($(\j.three)+(0,-0.08)$);
    }
    \path ($(0.north west)+(0.2,0.08)$) edge[bend left=20] ($({\addby{link}{1}}.north east)+(-0.3,0.08)$);
    \path ($({\addby{link}{1}}.south east)+(-0.2,-0.08)$) edge[bend left=50] ($(0.south west)+(0.2,-0.08)$);
    \end{tikzpicture}   
}
%%%%%%%%%%%%%%%%%%%%%%%%%%%%%%%%%%%%%%%%%%%%%%%%%%%%%%%%%%%%%%%%
%%%%%%%CS single linklist%%%%%%%%%%%%%%%%%%%%%%%%%%%%%%%%%%%%%%%
%%%%param(1):set of values             
%%%%format:\cslinklist{1,2,3,4}; 
%%%%%%%%%%%%%%%%%%%%%%%%%%%%%%%%%%%%%%%%%%%%%%%%%%%%%%%%%%%%%%%%
\NewDocumentCommand{\cslinklist}{O{} m}{
  	\begin{tikzpicture}[every node/.style={align=center},linkn/.style={
            very thick, rectangle split, 
            rectangle split parts=3, draw, 
            rectangle split horizontal, minimum size=18pt,
            inner sep=5pt, text=black,rounded corners,
            rectangle split part fill={blue!20, red!20, blue!20}
            },
        ref/.style={very thick, rectangle split, 
            rectangle split parts=2, draw, 
            , minimum size=18pt,
            inner sep=5pt, text=black,rounded corners,
            rectangle split part fill={blue!20, red!20}}
            ,remember picture, *->,shorten >=1pt, auto, start chain=going right
      ]
      
	\clearlist{link}           

    \addtolist{link}{#2}
    
    \node[linkn, on chain,] (0) {/ \nodepart{second}Head \\ Size = \getsize{link}}; 
   
	%%place and draw of link node and respective ref value
	\foreach \x in {1,...,\getsize{link}}{	 
    \node[linkn,on chain] (\x) {\nodepart{second}Idx(\x)};
    \node[ref, below= of \x] (ref\x) {Ref(\x)-Value \nodepart{second}\getindex{link}{\x}};
    \path ($(\x.center)+(0,-0.12)$) edge (ref\x);
    }
    
    \node[linkn, on chain,] ({\addby{link}{1}}) {\nodepart{second}Tail \nodepart{three} /}; 

   \foreach \v [count=\w from 1] in {0,...,\getsize{link}}{
    \draw let \p1 = (\v.three), \p2 = (\v.center) in (\x1,\y2) -- (\w);
   }
    \end{tikzpicture}   
}
%%%%%%%%%%%%%%%%%%%%%%%%%%%%%%%%%%%%%%%%%%%%%%%%%%%%%%%%%%%%%%%%
%%%%%%%CS loop single linklist%%%%%%%%%%%%%%%%%%%%%%%%%%%%%%%%%%%%%%%
%%%%param(1):set of values             
%%%%format:\cslinklist{1,2,3,4}; 
%%%%%%%%%%%%%%%%%%%%%%%%%%%%%%%%%%%%%%%%%%%%%%%%%%%%%%%%%%%%%%%%
\NewDocumentCommand{\cslooplinklist}{O{} m}{
  	\begin{tikzpicture}[every node/.style={align=center},linkn/.style={
            very thick, rectangle split, 
            rectangle split parts=3, draw, 
            rectangle split horizontal, minimum size=18pt,
            inner sep=5pt, text=black,rounded corners,
            rectangle split part fill={blue!20, red!20, blue!20}
            },
        ref/.style={very thick, rectangle split, 
            rectangle split parts=2, draw, 
            , minimum size=18pt,
            inner sep=5pt, text=black,rounded corners,
            rectangle split part fill={blue!20, red!20}}
            ,remember picture, *->,shorten >=1pt, auto, start chain
      ]
      
	\clearlist{link}           

    \addtolist{link}{#2}
    
    \node[linkn, on chain,] (0) {/ \nodepart{second}Head \\ Size = \getsize{link}}; 
   
	%%place and draw of link node and respective ref value
	\foreach \x in {1,...,\getsize{link}}{	 
    \node[linkn,on chain] (\x) {\nodepart{second}Idx(\x)};
    \node[ref, below= of \x] (ref\x) {Ref(\x)-Value \nodepart{second}\getindex{link}{\x}};
    \path ($(\x.center)+(0,-0.12)$) edge (ref\x);
    }
    
    \node[linkn, on chain,] ({\addby{link}{1}}) {\nodepart{second}Tail \nodepart{three} /}; 

   \foreach \v [count=\w from 1] in {0,...,\getsize{link}}{
    \draw let \p1 = (\v.three), \p2 = (\v.center) in (\x1,\y2) -- (\w);
   }
\path ($({\addby{link}{1}}.north east)+(-0.3,0.08)$) edge[bend right=20] ($(0.north west)+(0.3,0.08)$);
    \end{tikzpicture}   
}
%%%%%%%%%%%%%%%%%%%%%%%%%%%%%%%%%%%%%%%%%%%%%%%%%%%%%%%%%%%%%%%%
%%%%%%%CS vertical single linklist%%%%%%%%%%%%%%%%%%%%%%%%%%%%%%%%%%%%%%%
%%%%param(1):set of values             
%%%%format:\cslinklist{1,2,3,4}; 
%%%%%%%%%%%%%%%%%%%%%%%%%%%%%%%%%%%%%%%%%%%%%%%%%%%%%%%%%%%%%%%%
\NewDocumentCommand{\csvlinklist}{O{} m}{
  	\begin{tikzpicture}[every node/.style={align=center},linkn/.style={
            very thick, rectangle split, 
            rectangle split parts=3, draw, 
            minimum size=18pt,
            inner sep=5pt, text=black,rounded corners,
            rectangle split part fill={blue!20, red!20, blue!20}
            },
        ref/.style={very thick, rectangle split, 
            rectangle split parts=2, draw, 
            , minimum size=18pt,
            inner sep=5pt, text=black,rounded corners,
            rectangle split part fill={blue!20, red!20}}
            ,remember picture, *->,shorten >=1pt, auto, start chain=going below
      ]
      
	\clearlist{link}           

    \addtolist{link}{#2}
    
    \node[linkn, on chain,] (0) {/ \nodepart{second}Head \\ Size = \getsize{link}}; 
   
	%%place and draw of link node and respective ref value
	\foreach \x in {1,...,\getsize{link}}{	 
    \node[linkn,on chain] (\x) {\nodepart{second}Idx(\x)   };
    \node[ref, right= of \x] (ref\x) {Ref(\x)-Value \nodepart{second}\getindex{link}{\x}};
    \path ($(\x.center)+(0.4,0)$) edge (ref\x);
    }
    
    \node[linkn, on chain,] ({\addby{link}{1}}) {\nodepart{second}Tail \nodepart{three} /}; 

   \foreach \v [count=\w from 1] in {0,...,\getsize{link}}{
    \draw (\v.three) -- (\w.one);
   }
    \end{tikzpicture}
   
}
%%%%%%%%%%%%%%%%%%%%%%%%%%%%%%%%%%%%%%%%%%%%%%%%%%%%%%%%%%%%%%%%
%%%%%%%CS doubly linklist%%%%%%%%%%%%%%%%%%%%%%%%%%%%%%%%%%%%%%%
%%%%param(1):set of values             
%%%%format:\csdlinklist{1,2,3,4}; 
%%%%%%%%%%%%%%%%%%%%%%%%%%%%%%%%%%%%%%%%%%%%%%%%%%%%%%%%%%%%%%%%
\NewDocumentCommand{\csvdlinklist}{O{} m}{
\begin{tikzpicture}[every node/.style={align=center},linkn/.style={
            very thick, rectangle split, 
            rectangle split parts=3, draw, 
            minimum size=18pt,
            inner sep=5pt, text=black,rounded corners,
            rectangle split part fill={blue!20, red!20, blue!20}
            },
        ref/.style={very thick, rectangle split, 
            rectangle split parts=2, draw, 
            , minimum size=18pt,
            inner sep=5pt, text=black,rounded corners,
            rectangle split part fill={blue!20, red!20}}
            ,remember picture, *->, auto, start chain=going below
      ]

	\clearlist{link}           
	
    \addtolist{link}{#2}
    
    \node[linkn, on chain,] (0) {/ \nodepart{second}Head \\ Size = \getsize{link}}; 
    
	%%place and draw of link node and respective ref value
	\foreach \x in {1,...,\getsize{link}}{	 
    \node[linkn,on chain] (\x) {\nodepart{second} Idx(\x)};
    \node[ref, right= of \x] (ref\x) {Ref(\x)-Value \nodepart{second}\getindex{link}{\x}};
    \path ($(\x.center)+(0.4,0)$) edge (ref\x);
    }
    
    \node[linkn, on chain,] ({\addby{link}{1}}) {\nodepart{second}Tail \nodepart{three} /}; 

  
    \foreach \v [count=\i from 2, count=\j from 0] in {1,...,\getsize{link}}{

     \path ($(\v.three)+(0,0.08)$) edge[bend left] ($(\i.one)+(0,0.08)$);
    \path ($(\v.one)+(0,-0.08)$) edge[bend left] ($(\j.three)+(0,-0.08)$);
    }
    
    \end{tikzpicture}   
} 
%%%%%%%%%%%%%%%%%%%%%%%%%%%%%%%%%%%%%%%%%%%%%%%%%%%%%%%%%%%%%%%%%%%%%%%%%%%%%%%%%%%%%%%%%%%%%%%%%%%
%%%%%%%Hashmap%%%%%%%%%%%%%%%%%%%%%%%%%%%%%%%%%%%%%%%%%%%%%%%%%%%%%%%%%%%%%%%%%%%%%%%%%%%%%%%%%%%%%
%%param(2): name, set of key value pairs;             %%
%%%format: \hashmapset{name}{{key,val},{key,val},...};%%
%%%%%%%%%%%%%%%%%%%%%%%%%%%%%%%%%%%%%%%%%%%%%%%%%%%%%%%%
\newlist{keys}
\newlist{vals}

\newcommand\hashmapset[3]{
\begin{tikzpicture}[remember picture,auto,->,thick,
  inner/.style={circle,draw=blue!50,fill=blue!20,inner sep=3pt},
  outer/.style={rectangle,rectangle split, rectangle split parts=3, draw=black, rounded corners, 						fill=red!40, text centered, inner sep=10pt, text=black}
  ]
%reset lists upon instantiation%%
	\clearlist{keys}           %%
	\clearlist{vals}           %%
%%%%%%%%%%%%%%%%%%%%%%%%%%%%%%%%%
	\addtolist{keys}{#2}
	\addtolist{vals}{#3}
  %%construct Hashmap
  
    \node[outer] (#1){#1 \nodepart{two}\tikz\node[draw, circle,fill=blue!20, label={[xshift=-1.0cm,yshift=-0.85cm]Size}]{\getsize{keys}};
    \nodepart{three}%
     \tikz \node(scope){
    \begin{tikzpicture}[start chain=1 going below]
    % Place key value nodes vertical
    \foreach \i in {1,...,\getsize{keys}}{ 
      \node [inner, on chain=1] (bk\i)  {};
      \node [inner, right=of bk\i] (bv\i) {};
    }   
    \end{tikzpicture}
    };%
    };%%
    
    
     %% Place key value pairs references
    \node[ left=1cm of scope] (keys){
    \begin{tikzpicture}[start chain=1 going below]
     %%get keys   
    \foreach \k in {1,...,\getsize{keys}}{                                 
     \node [on chain=1] (key\k)  { \getindex{keys}{\k} };
     }
     \end{tikzpicture}
     };
    
                                         
    \node[right= 1cm of scope](values){
    \begin{tikzpicture}[start chain=1 going below]
     %%get values   
    \foreach \v in {1,...,\getsize{keys}}{                                 
     \node [on chain=1] (val\v)  { \getindex{vals}{\v} };
     }
    \end{tikzpicture}
    };      
      
 %%connect nodes to references
 
 \foreach \j  in {1,...,\getsize{keys}}{
 	 \path (bk\j .center) edge (key\j);
     \path (bv\j .center) edge (val\j);
 }
 
 \end{tikzpicture} %% end
}
%%%%%%%%%%%%%%%%%%%%%%%%%%%%%%%%%%%%%%%%%%%%%%%%%%%%%%%%%%%%%%%%%%%%%%%%%%%%%%%%%%%%%%%%%%%%%%%%%%%
%%%%%%%Base Hashmap%%%%%%%%%%%%%%%%%%%%%%%%%%%%%%%%%%%%%%%%%%%%%%%%%%%%%%%%%%%%%%%%%%%%%%%%%%%%%%%%%%%%%
%%param(2): name, set of key value pairs;             %%
%%%format: \hashmapset{name}{{key,val},{key,val},...};%%
%%%%%%%%%%%%%%%%%%%%%%%%%%%%%%%%%%%%%%%%%%%%%%%%%%%%%%%%
\newcommand\bhashmap[3]{
\begin{tikzpicture}[remember picture,auto,->,thick,
  inner/.style={circle,draw=blue!50,fill=blue!20,inner sep=3pt},
  outer/.style={rectangle,rectangle split, rectangle split parts=3, draw=black, rounded corners, 						fill=red!40, text centered, inner sep=10pt, text=black}
  ]
%reset lists upon instantiation%%
	\clearlist{keys}           %%
	\clearlist{vals}           %%
%%%%%%%%%%%%%%%%%%%%%%%%%%%%%%%%%
	\addtolist{keys}{#2}
	\addtolist{vals}{#3}
  %%construct Hashmap
  
    \node[outer] (#1){#1 \nodepart{two}\tikz\node[draw, circle,fill=blue!20, label={[xshift=-1.0cm,yshift=-0.85cm]Size}]{\getsize{keys}};
    \nodepart{three} a     b};
    \let\listy\empty
    \foreach \y in {1,...,\getsize{keys}}{
    \gappto\listy{
    \tikz \node(key\y){\getindex{keys}{\y}}; \&
    \tikz \node(bk\y){}; \&
    \tikz \node(bv\y){}; \&
    \tikz \node(val\y){\getindex{vals}{\y}}; \& \\
    }
    }%
 \end{tikzpicture} %% end 
}
%%%%%%%%%%%%%%%%%%%%%%%%%%%%%%%%%%%%%%%%%%%%%%%%%%%%%%%%%%%%%%%%
%%%%%%%single linklist%%%%%%%%%%%%%%%%%%%%%%%%%%%%%%%%%%%%%%%
%%%%param(1):set of values             
%%%%format:\linklist{1,2,3,4}; 
%%%%%%%%%%%%%%%%%%%%%%%%%%%%%%%%%%%%%%%%%%%%%%%%%%%%%%%%%%%%%%%%
\NewDocumentCommand{\linklist}{O{} m}{
  	\begin{tikzpicture}[every node/.style={align=center}, links/.style={}
            ,remember picture, *->,shorten >=1pt, auto,
      ]
    
    \ifstrempty{#2}{ }
    %else
    {
    
	\clearlist{link}           
    \addtolist{link}{#2}
    \ifnumcomp{\getsize{link}}{=}{1}{ \tikz\node{\getindex{link}{1}};}%
   {
	%%place and draw of link node and respective ref value
	\foreach \x in {1,...,\getsize{link}}{	 
    \node[links,on chain=going right] (\x) {\begin{tikzpicture}\node{\getindex{link}{\x}};\end{tikzpicture}};
    }
    
   \foreach \v [count=\w from 2] in {1,...,\addby{link}{-1}}{
    \draw (\v.east) -- (\w.west);
   }
   }
   }
    \end{tikzpicture}
   
}
%%%%%%%%%%%%%%%%%%%%%%%%%%%%%%%%%%%%%%%%%%%%%%%%%%%%%%%%%%%%%%%%
%%%%%%%double linklist%%%%%%%%%%%%%%%%%%%%%%%%%%%%%%%%%%%%%%%
%%%%param(1):set of values             
%%%%format:\linklist{1,2,3,4}; 
%%%%%%%%%%%%%%%%%%%%%%%%%%%%%%%%%%%%%%%%%%%%%%%%%%%%%%%%%%%%%%%%
\NewDocumentCommand{\dlinklist}{O{} m}{
  	\begin{tikzpicture}[every node/.style={align=center}, links/.style={}
            ,remember picture, *->,shorten >=1pt, auto, 
      ]
    
    \ifstrempty{#2}{ }
    %else
    {
    
	\clearlist{link}           
    \addtolist{link}{#2}
    \ifnumcomp{\getsize{link}}{=}{1}{ \tikz\node{\getindex{link}{1}};}%
   {
	%%place and draw of link node and respective ref value
	\foreach \x in {1,...,\getsize{link}}{	 
    \node[links,on chain=going right] (\x) {\getindex{link}{\x}};
    }
    
   \foreach \v [count=\w from 2] in {1,...,\addby{link}{-1}}{

     \path ($(\v.east)+(0,0.06)$) edge[bend left] ($(\w.west)+(0,0.06)$);
    \path ($(\w.west)+(0,-0.06)$) edge[bend left] ($(\v.east)+(0,-0.06)$);
    }
   }
   }
    \end{tikzpicture}   
}
%%%%%%%%%%%%%%%%%%%%%%%%%%%%%%%%%%%%%%%%%%%%%%%%%%%%%%%%%%%%%%%%
%%%%%%%single linklist%%%%%%%%%%%%%%%%%%%%%%%%%%%%%%%%%%%%%%%
%%%%param(1):set of values             
%%%%format:\linklist{1,2,3,4}; 
%%%%%%%%%%%%%%%%%%%%%%%%%%%%%%%%%%%%%%%%%%%%%%%%%%%%%%%%%%%%%%%%
\NewDocumentCommand{\vlinklist}{O{} m}{
  	\begin{tikzpicture}[every node/.style={align=center}, links/.style={}
            ,remember picture, *->,shorten >=1pt, auto
      ]
    
    \ifstrempty{#2}{ }
    %else
    {
    
	\clearlist{link}           
    \addtolist{link}{#2}
    \ifnumcomp{\getsize{link}}{=}{1}{ \tikz\node{\getindex{link}{1}};}%
   {
	%%place and draw of link node and respective ref value
	\foreach \x in {1,...,\getsize{link}}{	 
    \node[links,on chain=going below] (\x) {\getindex{link}{\x}};
    }
    
   \foreach \v [count=\w from 2] in {1,...,\addby{link}{-1}}{
    \draw (\v.south) -- (\w.north);
   }
   }
   }
    \end{tikzpicture}  
}


%%%%%%%%%%%%%%%%%%%%%%%%%%%%%%%%%%%%%%%%%%%%%%%%%%%%%%%%%%%%%%%%
%%%%%%%vertical double linklist%%%%%%%%%%%%%%%%%%%%%%%%%%%%%%%%%%%%%%%
%%%%param(1):set of values             
%%%%format:\linklist{1,2,3,4}; 
%%%%%%%%%%%%%%%%%%%%%%%%%%%%%%%%%%%%%%%%%%%%%%%%%%%%%%%%%%%%%%%%
\NewDocumentCommand{\dvlinklist}{O{} m}{
  	\begin{tikzpicture}[every node/.style={align=center}, links/.style={}
            ,remember picture, *->,shorten >=1pt, auto
      ]
    
    \ifstrempty{#2}{ }
    %else
    {
    
	\clearlist{link}           
    \addtolist{link}{#2}
    \ifnumcomp{\getsize{link}}{=}{1}{ \tikz\node{\getindex{link}{1}};}%
   {
	%%place and draw of link node and respective ref value
	\foreach \x in {1,...,\getsize{link}}{	 
    \node[links,on chain=going below] (\x) {\getindex{link}{\x}};
    }
    
   \foreach \v [count=\w from 2] in {1,...,\addby{link}{-1}}{

     \path ($(\v.south)+(0.06,0)$) edge[bend left] ($(\w.north)+(0.06,0)$);
    \path ($(\w.north)+(-0.06,0)$) edge[bend left] ($(\v.south)+(-0.06,0)$);
    }
   }
   }
    \end{tikzpicture}   
}
%%%%%%%%%%%%%%%%%%%%%%%%%%%%%%%%%%%%%%%%%%%%%%%%%%%%%%%%%%%%%%%%%%%%%%%%%%%%%%%%%%%%%%%%%%%%%%%%%
%%%%Test!%%%%%%%%%%%%%%%%%%%%%%%%%%%%%%%%%%%%%%%%%%%

%%%CS versions
\begin{adjustbox}{width=0.5\paperwidth,center,keepaspectratio}
\cslinklist{a,b,c,d,e}
\end{adjustbox}
\begin{adjustbox}{width=0.5\paperwidth,center,keepaspectratio}
\csdlinklist{foo,boo,coo,koo,woo,hoo}
\end{adjustbox}
\begin{adjustbox}{width=0.5\paperwidth,center,keepaspectratio}
\cslooplinklist{a,b,c,d}
\end{adjustbox}
\begin{adjustbox}{width=0.5\paperwidth,center,keepaspectratio}
\csloopdlinklist{a,b,c,d}
\end{adjustbox}

%%%simple form
\begin{adjustbox}{width=0.5\paperwidth,center,keepaspectratio}
\linklist{a,b,c,d}
\end{adjustbox}
\begin{adjustbox}{width=0.5\paperwidth,center,keepaspectratio}
\dlinklist{a,b,c,d}
\end{adjustbox}




%%%%verticals
\begin{adjustbox}{height=0.5\paperheight,center,keepaspectratio}
\csvdlinklist{foo,boo,coo,koo,woo,hoo}
\end{adjustbox}
\begin{adjustbox}{height=0.5\paperheight,center,keepaspectratio}
\csvlinklist{a,b,c,d,e}
\end{adjustbox}
\begin{adjustbox}{height=0.5\paperheight,center,keepaspectratio}
\vlinklist{a,b,c,d}
\end{adjustbox}
\begin{adjustbox}{height=0.5\paperheight,center,keepaspectratio}
\dvlinklist{a,b,c,d}
\end{adjustbox}

%%%%%%%%%%%%%%%%%%%%%%%%%%%%%%%%
\begin{adjustbox}{width=0.5\paperwidth,center,keepaspectratio}
\hashmapset{HashMap}{key1,key2,key3}{value3,value2,value3}
\end{adjustbox}\\

%%matrix-arrays

\begin{adjustbox}{width=0.5\paperwidth,center,keepaspectratio}
\lrmatrix{1}{\linklist{1,a},\linklist{2,b},\linklist{3,c}}
\end{adjustbox}\\

\begin{adjustbox}{width=0.5\paperwidth,center,keepaspectratio}
\lrmatrix{3}{\vlinklist{1,a},\vlinklist{2,b},\vlinklist{3,c}}
\end{adjustbox}\\


\lrmatrix{5}{1,2,3,4,5,6,7,8,9,10} \\
\vmatrix{2}{1,2,3,4,5,6,7,8,9,10}\\
\lrsmatrix{3}{1,2,3,4,5,6,7,8,9,10}\\
\rlsmatrix{3}{1,2,3,4,5,6,7,8,9,10}\\
\vsmatrix{3}{1,2,3,4,5,6,7,8,9,10}\\


%%%%%%%%%%%%%%%%%%%%%%%%%%%%%%%%%%%%%%%%%%%%%%%%%%%%%%%%%%%%%%%%%%%%%%%%%%%%%%%%%%%%%%%%%%%%%%%%%%%%
\end{document}