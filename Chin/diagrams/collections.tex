\documentclass{article}
\usepackage[pdftex,active,tightpage]{preview}
\usepackage{expl3, xparse}% http://ctan.org/pkg/xparse
\usepackage{etoolbox}% http://ctan.org/pkg/etoolbox
\usepackage{ifthen}
\usepackage{adjustbox}

\PreviewEnvironment{tikzpicture}
\errorcontextlines 10000

\usepackage{tikz}
\usetikzlibrary{calc,shapes,shadows,shapes.multipart,chains,matrix,positioning,arrows.meta,arrows,
graphs,graphdrawing}
\usegdlibrary{layered}
\usetikzlibrary{calc,shapes.multipart,chains,arrows,positioning}


%%%%%%%%%%%%%%%%%%%%%%%%%%%%%%%%%%%%%%%%%%%%%%%%%%%%%%%%%%%%%%%%%%%%%%%%%%%%%%%%%%%%%%%%%%%%%%%%%%%%%%%%%%%%% Helpers %%%%%%%%%%%%%%%%%%%%%%%%%%%%%%%%%%%%%%%%%%%%%%%%%%%%%%%%%%%%%%%%%%%%%%%%%%%%%%%%%%%%%%%%%%%%%%%%%%%%%%%%%%%%%%%%%%%%%%%%%%%%%%%%%%%%%%%%%%%%%%%%%%%%%%%%%%%%%%%%%%%%%%%%%%%%%%%%%%%%%%%%%%%%


\ExplSyntaxOn

\seq_new:N \g_parsed_seq %%global tmp parsed sequence

\NewDocumentCommand{\addby}{m m}{%
 \int_eval:n {\seq_count:c {#1_seq} + (#2)}
}%

%%%%%%%%%%%%%%%%%%%%%%%%%%%%
%%%List Processors %%%%%%%%%
%%%%%%%%%%%%%%%%%%%%%%%%%%%%

\NewDocumentCommand{\newlist}{ m }
 {
  \seq_new:c { #1_seq }
 }
 
\NewDocumentCommand{\addtolist}{ O{,} m m }
 {
 \seq_gclear:N \g_parsed_seq %%clear tmp
 \seq_gset_split:Nnn \g_parsed_seq {#1} {#3} %%split list by delimiter, store in tmp seq
 
 %%map function: add each element in tmp in given seq
 \seq_map_inline:Nn \g_parsed_seq
  {
   \seq_gput_right:cn { #2_seq } { ##1 }
  }
 
 }
 
\NewDocumentCommand{\getsize}{m}
{
  \seq_count:c {#1_seq} %%value of list length 
}

\NewDocumentCommand{\getindex}{ m m }
 {
  \seq_item:cn { #1_seq } { #2 }
 }
\NewDocumentCommand{\setindex}{ m m m }
 {
  \cs_set:Npx #3 { \seq_item:cn { #1_seq } { #2 } }
 }
\NewDocumentCommand{\clearlist}{ m }
 {
  \seq_gclear:c { #1_seq }
 }
\ExplSyntaxOff
 
%%%%%%%%%%%%%%%%%%%%%%%%%%%%%%%%%%%%%%%%%%%%%%%%%%%%%%%%%%%%%%%%%%%%%%%%%%%%%%%%%%%%%%%%%%%%%%%%%%%%%%%%%%%%%%
\begin{document}

\NewDocumentCommand{\linklist}{O{} m}{
\begin{adjustbox}{width=0.8\paperwidth,center,keepaspectratio}
  	\begin{tikzpicture}[every node/.style={align=center},linkn/.style={
            very thick, rectangle split, 
            rectangle split parts=3, draw, 
            rectangle split horizontal, minimum size=18pt,
            inner sep=5pt, text=black,rounded corners,
            rectangle split part fill={blue!20, red!20, blue!20}
        },
        ref/.style={very thick, rectangle split, 
            rectangle split parts=2, draw, 
            , minimum size=18pt,
            inner sep=5pt, text=black,rounded corners,
            rectangle split part fill={blue!20, red!20}}
            ,remember picture, 
         ->,shorten >=1pt, auto, start chain, very thick
      ]
    \newlist{link}
    \addtolist{link}{#2}
    
    \node[linkn, on chain,] (0) {/ \nodepart{second}Head \\ Size = \getsize{link}}; 
   
	%%place and draw of link node and respective ref value
	\foreach \x in {1,...,\getsize{link}}{	 
    \node[linkn,on chain] (\x) {\nodepart{second}Indx(\x)};
    \node[ref, below= of \x] (ref\x) {Value-Ref(\x) \nodepart{second}\getindex{link}{\x} };
    \path ($(\x.center)+(0,-0.1)$) edge (ref\x);
    }
    
    \node[linkn, on chain,] ({\addby{link}{1}}) {\nodepart{second}Tail \nodepart{three} /}; 

   \ifthenelse {\equal{\detokenize{#1}}{\detokenize{single}}}{
   \foreach \v [count=\w from 1] in {0,...,\getsize{link}}{
    \draw[*->] let \p1 = (\v.three), \p2 = (\v.center) in (\x1,\y2) -- (\w);
    }
    }%%else doubly
    {
    \foreach \v [count=\i from 2, count=\j from 0] in {1,...,\getsize{link}}{

     \path (\v.three) edge[bend left] (\i.one);
    \path (\v.one) edge[bend left] (\j.three);
    }
    }
    \end{tikzpicture}
   \end{adjustbox}
}
%%%%%%%%%%%%%%%%%%%%%%%%%%%%%%%%%%%%%%%%%%%%%%%%%%%%%%%%%%%%%%%%%%%%%%%%%%%%%%%%%%%%%%%%%%%%%%%%%%%%

    
%%%%%%%%%%%%%%%%%%%%%%%%%%%%%%%%%%%%%%%%%%%%%%%%%%%%%%%%%%%%%%%%%%%%%%%%%%%%%%%%%%%%%%%%%%%%%%%%%%%%%%%
%%%%%%%Hashmap%%%%%%%%%%%%%%%%%%%%%%%%%%%%%%%%%%%%%%%%%%%%%%%%%%%%%%%%%%%%%%%%%%%%%%%%%%%%%%%%%%%%%%%%%%%%param(2): name, set of key value pairs;             %%%%%%%%%%%%%%%%%%%%%%%%%%%%%%%%%%%%%%%%%%%%%%%%
%%%format: \hashmapset{name}{{key,val},{key,val},...}; %%%%%%%%%%%%%%%%%%%%%%%%%%%%%%%%%%%%%%%%%%%%%%%%
%%%%%%%%%%%%%%%%%%%%%%%%%%%%%%%%%%%%%%%%%%%%%%%%%%%%%%%%%%%%%%%%%%%%%%%%%%%%%%%%%%%%%%%%%%%%%%%%%%%%%%%
\newcommand\hashmapset[3]{
\begin{adjustbox}{width=0.8\paperwidth,center,keepaspectratio}
\begin{tikzpicture}[remember picture, auto,->,thick,
  inner/.style={circle,draw=blue!50,fill=blue!20,inner sep=3pt},
  outer/.style={rectangle,rectangle split, rectangle split parts=2, draw=black, rounded corners, 						fill=red!40, text centered, inner sep=10pt, text=black}
  ]
	\newlist{keys}
	\addtolist{keys}{#2}
	\newlist{vals}
	\addtolist{vals}{#3}
  %%construct Hashmap
  \node[outer] (#1) {#1
    
    \nodepart{two}
    \begin{tikzpicture}[start chain= going below]
    %% Place key value nodes vertical
    \foreach \i in {1,...,\getsize{keys}}{ 
      \node [inner, on chain] (bk\i)  {};
      \node [inner, right=of bk\i] (bv\i) {};
    }  
    \end{tikzpicture}
    }; %
    
    
 %% Place key value pairs references
    \foreach \j in {1,...,\getsize{keys}}{   
      
    %%get key-value pair     
                                        
     \node [left= 1cm of bk\j] (key\j)  { \getindex{keys}{\j} };
                                         
     \node [right= 1cm of bv\j] (val\j)  { \getindex{vals}{\j}};
}      
      
 %%connect nodes to references
 
 \foreach \k  in {1,...,\getsize{keys}}{
 	 \path (bk\k .center) edge (key\k);
     \path (bv\k .center) edge (val\k);
 }
 
 \end{tikzpicture} %% end
 \end{adjustbox}
}
%%%%Test!%%%%%%%%%%%%%%%%%%%%%%%%%%%%%%%%%%%%%%%%%%%%%%%%%%%%%%%%%%%%%%%%%%%%%%%%%%%%%%%%%%%%%%%%%%%%%%
\hashmapset{name}{ky1,ky2,ky3}{v1,v2,v3}

\linklist[single]{3,4,5,6,7,8,{drrt,hgfk},tufik,j0}


%%%%%%%%%%%%%%%%%%%%%%%%%%%%%%%%%%%%%%%%%%%%%%%%%%%%%%%%%%%%%%%%%%%%%%%%%%%%%%%%%%%%%%%%%%%%%%%%%%%%
\end{document}