% Class Diagram Generator
% Author: Jovaughn Chin
\documentclass{minimal}
\usepackage[a4paper,margin=1cm,landscape]{geometry}
\usepackage{tikz}
\usetikzlibrary{calc,shapes,shadows,shapes.multipart,chains,matrix,positioning,arrows.meta,arrows}
\usepackage{xparse}% http://ctan.org/pkg/xparse
\usepackage{etoolbox}% http://ctan.org/pkg/etoolbox
\usepackage{ifthen}
\errorcontextlines 10000
%%%%%%%%%%%%%%%%%%%%%%%%%%%%%%%%%%%%%%%%%%%%%%%%%%%%%%%%%%%%%%%%%%%%%%%%%%%%%%%%%%%%%%%%%%%%%%%%%%%%%%%
\DeclareListParser{\ListParsecomma}{,}
\DeclareListParser{\ListParsesemi}{;}
\DeclareListParser{\ListParsecolon}{:}
\newcounter{cnt}
\newcounter{xcnt}

\def\addby#1#2{%
\pgfmathsetmacro{\result}{#1}% 
\pgfmathtruncatemacro{\result}{\result+ #2}% 
\global\let\result\result
}%
\def\subby#1#2{%
\pgfmathsetmacro{\sresult}{#1}% 
\pgfmathtruncatemacro{\sresult}{\sresult- #2}% 
\global\let\sresult\sresult
}%
%%%%%Semi colon parser %%%%%%%%%%%%%%%%%%%%%%%%%%%%%%%%%%%%
\def\ygetarray#1{%
\pgfmathsetmacro{\ylenarray}{0}% 

\foreach \i in {#1}{%
\pgfmathtruncatemacro{\ylenarray}{\ylenarray+1}% 
\global\let\ylenarray\ylenarray
}%
\pgfmathsetmacro{\ylenminus}{0}%
\pgfmathtruncatemacro{\ylenminus}{\ylenarray-1}%  
\global\let\ylenminus\ylenminus

\setcounter{ycnt}{0}
\newcommand\ysetval[2]{%
  \csdef{list##1}{##2}}
\newcommand\yaddval[1]{%
  \stepcounter{ycnt}%
  \csdef{list\theycnt}{##1}}
\newcommand\ygetval[1]{%
  \csuse{list##1}}
  
  \renewcommand{\do}[1]{\yaddval{##1}}
  \expandafter\expandafter\expandafter\ListParsesemi\expandafter\expandafter\expandafter{#1}
}
%%%%%% colon parser %%%%%%%%%%%%
\def\xgetarray#1{%
\pgfmathsetmacro{\xlenarray}{0}% 
 
\foreach \i in {#1}{%
\pgfmathtruncatemacro{\xlenarray}{\xlenarray+1}% 
\global\let\xlenarray\xlenarray
}%
\pgfmathsetmacro{\xlenminus}{0}%
\pgfmathtruncatemacro{\xlenminus}{\xlenarray-1}%  
\global\let\xlenminus\xlenminus

\setcounter{xcnt}{0}
\newcommand\xsetval[2]{%
  \csdef{list##1}{##2}}
\newcommand\xaddval[1]{%
  \stepcounter{xcnt}%
  \csdef{list\thexcnt}{##1}}
\newcommand\xgetval[1]{%
  \csuse{list##1}}
  
  \renewcommand{\do}[1]{\xaddval{##1}}
  \expandafter\expandafter\expandafter\ListParsecolon\expandafter\expandafter\expandafter{#1}
}
%%%%%%%%% comma parser %%%%%%%%%%%%%%%%%%%
\def\getarray#1{%
\pgfmathsetmacro{\lenarray}{0}% 

\foreach \i in {#1}{%
\pgfmathtruncatemacro{\lenarray}{\lenarray+1}% 
\global\let\lenarray\lenarray
}%
\pgfmathsetmacro{\lenminus}{0}%
\pgfmathtruncatemacro{\lenminus}{\lenarray-1}%  
\global\let\lenminus\lenminus

\setcounter{cnt}{0}
\newcommand\setval[2]{%
  \csdef{list##1}{##2}}
\newcommand\addval[1]{%
  \stepcounter{cnt}%
  \csdef{list\thecnt}{##1}}
\newcommand\getval[1]{%
  \csuse{list##1}}
  
  \renewcommand{\do}[1]{\addval{##1}}
  \expandafter\expandafter\expandafter\ListParsecomma\expandafter\expandafter\expandafter{#1}
}
%%%%%%%%%%%%%%%%%%%%%%%%%%%%%%%%%%%%%%%%%%%%%%%%%%%%%%%%%%%%%%%%%%%%%%%%%%%%%%%%%%%%%%%%%%%%%%%%%%%%%%%
\begin{document}
\tikzstyle{abstract}=[rectangle, draw=black, rounded corners, fill=blue!40, drop shadow,
        text centered, anchor=north, text=white, text width=3cm]
\tikzstyle{comment}=[rectangle, draw=black, rounded corners, fill=green, drop shadow,
        text centered, anchor=north, text=white, text width=3cm]
\tikzstyle{myarrow}=[->, >=open triangle 45, thick]
%%%%%%%%%%%%%%%%%%%%%%%%%%%%%%%%%%%%%%%%%%%%%%%%%%%%%%%%%%%%%%%%%%%%%%%%%%%%%%%%%%%%%%%%%%%%%%%%%%%%%%%%%Class style   
\tikzstyle{class}=[rectangle,rectangle split, rectangle split parts=2, draw=black, rounded corners, fill=red!40, drop shadow, text centered, anchor=north, text=white, text width=3cm]
%%%%%%%%%%%%%%%%%%%%%%%%%%%%%%%%%%%%%%%%%%%%%%%%%%%%%%%%%%%%%%%%%%%%%%%%%%%%%%%%%%%%%%%%%%%%%%%%%%%%%%%%%%Relationship arrow styles%%%%%%%%%%%%%%%%%%%%%%%%%%%%%%%%%%%%%%%%%%%%%%%%%%%%%%%%%%%%%
\tikzstyle{generalization}=[->, >=open triangle 45, thick]
\tikzstyle{dependency}=[dashed,->, >=angle 60, thick]
\tikzstyle{aggregation}=[->, >=open diamond, thick]
\tikzstyle{composition}=[->, >=diamond, thick]
\tikzstyle{realization}=[dashed,->, >=triangle 45, thick]
\tikzstyle{line}=[-, thick]
\tikzstyle{dash}=[dashed,-, thick]
%%%%%%%%%%%%%%%%%%%%%%%%%%%%%%%%%%%%%%%%%%%%%%%%%%%%%%%%%%%%%%%%%%%%%%%%%%%%%%%%%%%%%%%%%%%%%%%%%%%%%%%%%%%% Fiddle sample %%%%%%%%%%%%%%%%%%%%%%%%%%%%%%%%%%%%%%%%%%%%%%%%%%%%%%%%%%%%%%%%%%%
\begin{center}
\begin{tikzpicture}[node distance=2cm]
    \node (Item) [abstract, rectangle split, rectangle split parts=2]
        {
            \textbf{ITEM}
            \nodepart{second}name
        };
    \node (AuxNode01) [text width=4cm, below=of Item] {};
    \node (Component) [abstract, rectangle split, rectangle split parts=2, left=of AuxNode01]
        {
            \textbf{COMPONENT}
            \nodepart{second}nil
        };
    \node (System) [abstract, rectangle split, rectangle split parts=2, right=of AuxNode01]
        {
            \textbf{SYSTEM}
            \nodepart{second}parts
        };
    \node (AuxNode02) [text width=0.5cm, below=of Component] {};
    \node (Sensor) [abstract, rectangle split, rectangle split parts=2, left=of AuxNode02]
        {
            \textbf{SENSOR}
            \nodepart{second}nil
        };
    \node (Part) [abstract, rectangle split, rectangle split parts=2, right=of AuxNode02]
        {
            \textbf{PART}
            \nodepart{second}nil
        };
        
    \node (AuxNode03) [below=of Sensor] {};
    \node (Pressure) [abstract, rectangle split, rectangle split parts=2, left=of AuxNode03, xshift=2cm]
        {
            \textbf{Pressure}
            \nodepart{second}nil
        };
    \node (Temperature) [abstract, rectangle split, rectangle split parts=2, right=of AuxNode03, xshift=-2cm]
        {
            \textbf{Temperature}
            \nodepart{second}nil
        };
    \node (PressureInstants) [comment, rectangle split, rectangle split parts=2, below=0.2cm of Pressure, text justified]
        {
            \textbf{Instants}
            \nodepart{second}fw-p-suction\newline fw-p-delivery\newline fw-p-loop\newline sw-p-suction\newline sw-p-delivery
                \newline sw-p-loop
        };
    \node (ClOp) [abstract, rectangle split, rectangle split parts=2, below=0.4cm of PressureInstants]
        {
            \textbf{Closed/Open}
            \nodepart{second}nil
        };
    \node (ClOpInstants) [comment, rectangle split, rectangle split parts=2, below=0.2cm of ClOp, text justified]
        {
            \textbf{Instants}
            \nodepart{second}fw-clop-warm-up\newline sw-clop-control
        };
    \node (TemperatureInstants) [comment, rectangle split, rectangle split parts=2, below=0.2cm of Temperature, text justified]
        {
            \textbf{Instants}
            \nodepart{second}fw-t-engine\newline fw-t-heat-exch.\newline sw-t-heat-exch.
        };
    \node (Level) [abstract, rectangle split, rectangle split parts=2, below=0.4cm of TemperatureInstants]
        {
            \textbf{Level}
            \nodepart{second}nil
        };
    \node (LevelInstants) [comment, rectangle split, rectangle split parts=2, below=0.2cm of Level, text justified]
        {
            \textbf{Instants}
            \nodepart{second}fw-l-tank
        };
    \node (Ammeter) [abstract, rectangle split, rectangle split parts=2, below=0.4cm of LevelInstants]
        {
            \textbf{Ammeter}
            \nodepart{second}nil
        };
    \node (AmmeterInstants) [comment, rectangle split, rectangle split parts=2, below=0.2cm of Ammeter, text justified]
        {
            \textbf{Instants}
            \nodepart{second}fw-pump-ammeter\newline sw-pump-ammeter
        };
        
    \node (AuxNode04) [below=of Part] {};
    \node (Pump) [abstract, rectangle split, rectangle split parts=2, left=of AuxNode04, xshift=2cm]
        {
            \textbf{Pump}
            \nodepart{second}nil
        };
    \node (Valve) [abstract, rectangle split, rectangle split parts=2, right=of AuxNode04, xshift=-2cm]
        {
            \textbf{Valve}
            \nodepart{second}nil
        };
    \node (PumpInstants) [comment, rectangle split, rectangle split parts=2, below=0.2cm of Pump, text justified]
        {
            \textbf{Instants}
            \nodepart{second}fw-pump\newline sw-pump
        };
    \node (Tank) [abstract, rectangle split, rectangle split parts=2, below=0.4cm of PumpInstants]
        {
            \textbf{Tank}
            \nodepart{second}nil
        };
    \node (ValveInstants) [comment, rectangle split, rectangle split parts=2, below=0.2cm of Valve, text justified]
        {
            \textbf{Instants}
            \nodepart{second}fw-suction-valve\newline fw-delivery-valve\newline sw-suction-valve\newline sw-delivery-valve
                \newline sw-discharge-valve\newline sw-control-valve
        };
    \node (Engine) [abstract, rectangle split, rectangle split parts=2, below=0.4cm of ValveInstants]
        {
            \textbf{Engine}
            \nodepart{second}nil
        };
    \node (TankInstants) [comment, rectangle split, rectangle split parts=2, below=0.2cm of Tank, text justified]
        {
            \textbf{Instants}
            \nodepart{second}fw-expansion-tank
        };
    \node (HeatExchanger) [abstract, rectangle split, rectangle split parts=2, below=0.4cm of TankInstants]
        {
            \textbf{Heat Exchanger}
            \nodepart{second}nil
        };
    \node (HeatExchangerInstants) [comment, rectangle split, rectangle split parts=2, below=0.2cm of HeatExchanger, text justified]
        {
            \textbf{Instants}
            \nodepart{second}fw-heat-exchanger
        };
    \node (EngineInstants) [comment, rectangle split, rectangle split parts=2, below=0.2cm of Engine, text justified]
        {
            \textbf{Instants}
            \nodepart{second}fw-engine
        };
    \node (Strainer) [abstract, rectangle split, rectangle split parts=2, below=0.4cm of HeatExchangerInstants]
        {
            \textbf{Strainer}
            \nodepart{second}nil
        };
    \node (StrainerInstants) [comment, rectangle split, rectangle split parts=2, below=0.2cm of Strainer, text justified]
        {
            \textbf{Instants}
            \nodepart{second}sw-strainer
        };
    \node (Coolant) [abstract, rectangle split, rectangle split parts=2, below=0.4cm of EngineInstants]
        {
            \textbf{Coolant}
            \nodepart{second}nil
        };
    \node (CoolantInstants) [comment, rectangle split, rectangle split parts=2, below=0.2cm of Coolant, text justified]
        {
            \textbf{Instants}
            \nodepart{second}fw-coolant\newline sw-coolant
        };  

    \node (AuxNode05) [below=of System] {};
    \node (CoolingSystem) [abstract, rectangle split, rectangle split parts=2, left=of AuxNode05, xshift=2cm]
        {
            \textbf{Cooling System}
            \nodepart{second}nil
        };
    \node (CoolingLoop) [abstract, rectangle split, rectangle split parts=2, right=of AuxNode05, xshift=-2cm]
        {
            \textbf{Cooling Loop}
            \nodepart{second}nil
        };
    \node (CoolingSystemInstants) [comment, rectangle split, rectangle split parts=2, below=0.2cm of CoolingSystem, text justified]
        {
            \textbf{Instants}
            \nodepart{second}cool
        };
    \node (CoolingLoopInstants) [comment, rectangle split, rectangle split parts=2, below=0.2cm of CoolingLoop, text justified]
        {
            \textbf{Instants}
            \nodepart{second}fw-loop\newline sw-loop
        };
    
    \draw[myarrow] (Component.north) -- ++(0,0.8) -| (Item.south);
    \draw[line] (Component.north) -- ++(0,0.8) -| (System.north);
    
    \draw[myarrow] (Sensor.north) -- ++(0,0.8) -| (Component.south);
    \draw[line] (Sensor.north) -- ++(0,0.8) -| (Part.north);
    
    \draw[line] (Pressure.west) -- ++(-0.2,0);
    \draw[line] (Temperature.east) -- ++(0.2,0);
    \draw[line] (Level.east) -- ++(0.2,0);
    \draw[myarrow] (ClOp.west) -- ++(-0.2,0) -- ([yshift=0.9cm, xshift=-0.2cm] Pressure.north west) -|
     ([xshift=-1cm]Sensor.south);
    \draw[myarrow] (Ammeter.east) -- ++(0.2,0) -- ([yshift=0.5cm, xshift=0.2cm] Temperature.north east) -|
     ([xshift=1cm]Sensor.south);
     
    \draw[line] (Tank.west) -- ++(-0.2,0);
    \draw[line] (HeatExchanger.west) -- ++(-0.2,0);
    \draw[line] (Pump.west) -- ++(-0.2,0);
    \draw[line] (Valve.east) -- ++(0.2,0);
    \draw[line] (Engine.east) -- ++(0.2,0);
    \draw[myarrow] (Strainer.west) -- ++(-0.2,0) -- ([yshift=0.5cm, xshift=-0.2cm] Pump.north west) -|
     ([xshift=-1cm]Part.south);
    \draw[myarrow] (Coolant.east) -- ++(0.2,0) -- ([yshift=0.5cm, xshift=0.2cm] Valve.north east) -|
     ([xshift=1cm]Part.south);
     
    \draw[myarrow] (CoolingSystem.north) -- ++(0,0.8) -| (System.south);
    \draw[line] (CoolingSystem.north) -- ++(0,0.8) -| (CoolingLoop.north);
        
        
\end{tikzpicture}
\end{center}
%%%%%%%%%%%%%%%%%%%%%%%%%%%%%%%%%%%%%%%%%%%%%%%%%%%%%%%%%%%%%%%%%%%%%%%%%%%%%%%%%%%%%%%%%%%%%%%%%%%%%%%%%%%Automate Construct Class diagram %%%%%%%%%%%%%%%%%%%%%%%%%%%%%%%%%%%%%%%%%%%%%%%%%%%%%%%%%%%%%%%%
%%%%%%%%%%%%%%%%%%%%%%%%%%%%%%%%%%%%%%%%%%%%%%%%%%%%%%%%%%%%%%%%%%%%%%%%%%%%%%%%%%%%%%%%%%%%%%%%%%%%%%%%%%%%%%Define class %%%%%%%%%%%%%%%%%%%%%%%%%%%%%%%%%%%%%%%%%%%%%%%%%%%%%%%%%%%%%%%%%%%%%%%%%%%%%%
\newcommand{\class}[2]{
\node[class] (#1) {\textbf #1 \nodepart{second}#2};
}
%%%%%%%draw class%%%%%%%%
\newcommand{\drawclass}[2]{
\tikz \node[class] (#1) {\textbf #1 \nodepart{second}#2};
}
\drawclass{name}{content}

%%%%%%%draw class diagram%%%%%%%%%%%%%%%%%%%%%%%%%%%%%%%
\newcommand{\classdiagram}[1]{
%parse input of list of class relationships(parsed by semi-colon)
\getarray{#1}

\foreach \r in {1,...,\lenarray}{
%parse individual relationship(parsed by colon)
\xgetarray{\getval{1}}
%parse relationship children(parsed by comma)
\ygetarray{\xgetval{3}}

%%%%%%check relationship type%%%%%%%%%%%%%%%%%%%%%%%%
\ifx\xgetval{2}g
\else \ifx\xgetval{2}c
\else \ifx\xgetval{2}d
\else \ifx\xgetval{2}di
\else \ifx\xgetval{2}da
\else \ifx\xgetval{2}dp
\else \ifx\xgetval{2}dr
\else \ifx\xgetval{2}dd
\else \ifx\xgetval{2}r
\else \ifx\xgetval{2}s
\else \ifx\xgetval{2}t
\else \ifx\xgetval{2}u
\else \ifx\xgetval{2}m
\else \ifx\xgetval{2}i
\else \ifx\xgetval{2}b
\else \ifx\xgetval{2}a

}
}
%%%%%%%%%%%%%%%%%%%%%%%%%%%%%%%%%%%%%%%%%%%%%%%%%%%%%%%%%%%%%%%%%%%%%%%%%%%%%%%%%%%%%%%%%%%%%%%%%%%%

\end{document}