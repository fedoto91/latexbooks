\documentclass{article}
\usepackage{collection}

\begin{document}


\begin{center}
{\Huge Collection Package}
\end{center}
\indent \textbf{Objective:} is to aid in the visualization of Data Structures and Objects in the Theory of Computer Science, as well as, being flexible enough to accept any object(text,pictures,or any graphics) as input and customize to the user's desire.
\newpage
%%%CS versions
\begin{center}
\textbf{{\Large Computer Science Data Structures: Linked Lists}}
\end{center}

CS Singly Linked List\\
Format:{\textbackslash}cslinklist[Option:styling]{\{}objects{\}}\\
Example:{\textbackslash}cslinklist{\{}a,b,c,d,e{\}} \newline \\
\begin{adjustbox}{width=0.5\paperwidth,center,keepaspectratio}
\cslinklist{a,b,c,d,e}
\end{adjustbox}\\
CS Doubly Linked List\\
Format:{\textbackslash}csdlinklist[Option:styling]{\{}objects{\}}\\
Example:{\textbackslash}csdlinklist{\{}foo,boo,coo,koo,woo,hoo{\}} \newline \\
\begin{adjustbox}{width=0.5\paperwidth,center,keepaspectratio}
\csdlinklist{foo,boo,coo,koo,woo,hoo}
\end{adjustbox}\\
CS Circular Singly Linked List\\
Format:{\textbackslash}cslooplinklist[Option:styling]{\{}objects{\}}\\
Example:{\textbackslash}cslooplinklist{\{}a,b,c,d{\}} \newline \\
\begin{adjustbox}{width=0.5\paperwidth,center,keepaspectratio}
\cslooplinklist{a,b,c,d}
\end{adjustbox}\\
CS Circular Doubly Linked List\\
Format:{\textbackslash}csloopdlinklist[Option:styling]{\{}objects{\}}\\
Example:{\textbackslash}csloopdlinklist{\{}a,b,c,d{\}} \newline \\
\begin{adjustbox}{width=0.5\paperwidth,center,keepaspectratio}
\csloopdlinklist{a,b,c,d}
\end{adjustbox}\\
%%%simple form
\newpage
\begin{center}
\textbf{{\Large Simple collections: Linked Lists}}
\end{center}
Simple Singly Linked List\\
Format:{\textbackslash}linklist[Option:styling]{\{}objects{\}}\\
Example:{\textbackslash}linklist{\{}a,b,c,d{\}} \newline \\
\begin{adjustbox}{width=0.5\paperwidth,center,keepaspectratio}
\linklist{a,b,c,d}
\end{adjustbox}\\

Example:{\textbackslash}linklist{\{}{\{}{\textbackslash}tikz{\textbackslash}node[draw,circle]{\{}0{\}};{\}},{\{}{\textbackslash}tikz{\textbackslash}node[draw,circle]{\{}gh{\}};{\}},{\{}{\textbackslash}tikz{\textbackslash}node[draw,circle]{\{}b{\}};{\}},{\{}{\textbackslash}tikz{\textbackslash}node[draw,circle]{\{}c{\}};{\}}{\}}\\
\begin{adjustbox}{width=0.5\paperwidth,center,keepaspectratio}
\linklist{{\tikz\node[draw,circle]{0};},{\tikz\node[draw,circle]{gh};},{\tikz\node[draw,circle]{b};},{\tikz\node[draw,circle]{c};}} 
\end{adjustbox}\\


Simple Doubly Linked List\\
Format:{\textbackslash}dlinklist[Option:styling]{\{}objects{\}}\\
Example:{\textbackslash}dlinklist{\{}a,b,c,d{\}} \newline \\
\begin{adjustbox}{width=0.5\paperwidth,center,keepaspectratio}
\dlinklist{a,b,c,d}
\end{adjustbox}\\
Simple loop Linked List\\
Format:{\textbackslash}looplinklist[Option:styling]{\{}objects{\}}\\
Example:{\textbackslash}looplinklist{\{}a,b,c,d{\}} \newline \\
\begin{adjustbox}{width=0.5\paperwidth,center,keepaspectratio}
\looplinklist{a,b,c,d}
\end{adjustbox}\\
Simple Doubly Linked List\\
Format:{\textbackslash}loopdlinklist[Option:styling]{\{}objects{\}}\\
Example:{\textbackslash}loopdlinklist{\{}a,b,c,d{\}} \newline \\
\begin{adjustbox}{width=0.5\paperwidth,center,keepaspectratio}
\loopdlinklist{a,b,c,d}
\end{adjustbox}
\newpage
\begin{center}
\textbf{{\Large Option(Vertical): CS vs Simple: Singly and Doubly Linked Lists}}
\end{center}

\begin{center}
\begin{adjustbox}{height=0.6\paperheight,width=\paperwidth,center,keepaspectratio}
\lrmatrix{4}{{{\textbackslash}csvdlinklist{\{}foo,boo,koo{\}}},{{\textbackslash}csvlinklist{\{}a,b,c{\}}},{{\textbackslash}vlinklist{\{}a,b,c,d,e,f,g,h,i{\}}},{{\textbackslash}dvlinklist{\{}a,b,c,d,e,f,g,h,i{\}}},\csvdlinklist{foo,boo,koo},\csvlinklist{a,b,c},\vlinklist{a,b,c,d,e,f,g,h,i},\vdlinklist{a,b,c,d,e,f,g,h,i}}
\end{adjustbox}
\end{center}

%\begin{adjustbox}{width=0.8\paperwidth,center,keepaspectratio}
%\hashmapset{namew}{1,2,3}{\cslinklist[{scale=0.6,every node/.style={transform shape}}]{a,c,d},%\csdlinklist[scale=0.6, every node/.style={transform shape}]{a,c,d},hello}
%\end{adjustbox}
%%%%%%%%%%%%%%%%%%%%%%%%%%%%%%%%%%%%%%%%%%%
%%%%%%%%%%%%%%%%%%%%%%%%%%%%%%%%%%%%%%


%%%%%%%%%%%%%%%%%%%%%%%%%%%%%%%%%%%%%%%%%%%%%%%%
\newpage
\begin{center}
\textbf{{\Large Data Structure: HashMap Object}}
\end{center}
Format:{\textbackslash}hashmapset[Option:styling]{\{}Name of Hashmap{\} }{\{}list of Keys{\}}{\{}list of Values{\}}\\
Example:{\textbackslash}hashmapset{\{}HashMap{\} }{\{}key1,key2,key3{\}}{\{}value1,value2,value3{\}}\newline \\
\begin{adjustbox}{width=0.5\paperwidth,center,keepaspectratio}
\hashmapset{HashMap}{key1,key2,key3}{value1,value2,value3}
\end{adjustbox}

\newpage
\begin{center}
\textbf{{\Large Data Structure: Stacks}}
\end{center}

\section*{Vertical Stack}
{\textbackslash}begin{\{}stack{\}}{\{}vertical{\}}
  {\textbackslash}cell[blue]{\{}cell 1{\}}{\{}1{\}}
{\textbackslash}end{\{}stack{\}} \\ \newline
\begin{stack}{vertical}
  \cell[blue]{cell 1}{1}
\end{stack}\\ \newline

\section*{Horizontal Stack}
{\textbackslash}begin{\{}stack{\}}{\{}horizontal{\}}
  {\textbackslash}cell[blue]{\{}cell 1{\}}{\{}1{\}}
{\textbackslash}end{\{}stack{\}} \\ \newline
\begin{stack}{horizontal}
  \cell[blue]{cell 1}{1}
  \cell[red]{cell 2}{2}
\end{stack}

\newpage
\begin{center}
\textbf{{\Large ARRAYS:}}
\end{center}
(Left-Right)\\ 
Format:{\textbackslash}lrmatrix{\{}number of columns to split list of objects{\} }{\{}objects{\}}\\
Example1:{\textbackslash}lrmatrix{\{}4{\} }{\{}1,2,3,4,5,6,7,a,b,c{\}}\\
Example2:{\textbackslash}lrmatrix{\{}2{\} }{\{}{\textbackslash}includegraphics{\{}ex1.jpg{\}},{\textbackslash}includegraphics{\{}ex2.jpg{\}},\\{\textbackslash}includegraphics{\{}ex3.jpg{\}},{\textbackslash}includegraphics{\{}ex4.jpg{\}}{\}}\newline \\

\lrmatrix{4}{1,2,3,4,5,6,7,a,b,c} \lrmatrix{2}{\includegraphics[width=30mm]{computer_science_zoomed_by_blackblood357-d37ofou.jpg},\includegraphics[width=30mm]{computer-science1.jpg},\includegraphics[width=30mm]{computer-science1.jpg},\includegraphics[width=30mm]{computer_science_zoomed_by_blackblood357-d37ofou.jpg}} \newline \\

(Vertical)\\
Format:{\textbackslash}vmatrix{\{}number of rows to split list of objects{\}}{\{}objects{\}}\\
Example:{\textbackslash}vmatrix{\{}2{\}}{\{}1,2,3,4,5,6,7,a,b,c{\}}\\
\vmatrix{2}{1,2,3,4,5,6,7,a,b,c}
\newpage
\textbf{Data Structure:(Simple Hashtable)}\\
{\textbackslash}lrmatrix{\{}1{\} }{\{}\\{\textbackslash}linklist{\{}{\{}{\textbackslash}tikz{\textbackslash}node[draw,circle]{\{}0{\}};{\}},a,b,c{\}},\\{\textbackslash}linklist{\{}{\{}{\textbackslash}tikz{\textbackslash}node[draw,circle]{\{}1{\}};{\}},d,e,f{\}},\\{\textbackslash}linklist{\{}{\{}{\textbackslash}tikz{\textbackslash}node[draw,circle]{\{}0{\}};{\}},g,h,i{\}}{\}}\newline \\
\lrmatrix{1}{\linklist{{\tikz\node[draw,circle]{0};},a,b,c},\linklist{{\tikz\node[draw,circle]{1};},d,e,f},\linklist{{\tikz\node[draw,circle]{2};},g,h,i}}


\section*{Diagrams}
Format:\\
{\textbackslash}begin{\{}diagram{\}}{\{}Styling{\}}\\ 
-----Content------\\
{\textbackslash}end{\{}diagram{\}}\\
\subsection*{Example: Class Diagram}
Format:\\
{\textbackslash}class[Option: styling]{\{}class var{\}}[Option:Title][Option:members][Option:operations]\\
{\textbackslash}rel{\{}relationship{\}}{\{}derive source{\}} {\{}deriver{\}}\\
EX:
{\textbackslash}begin{\{}diagram{\}}{\{}node distance=5cm,layered layout{\}}\\
{\textbackslash}class[rounded corners, fill=red!20]{\{}a{\}}[Interface{\textbackslash}{\textbackslash}Classes][members][operations]\\
{\textbackslash}class[rounded corners, fill=blue!20]{\{}b{\}}[class1][members][operations]\\
{\textbackslash}class{\{}c{\}}[class2][members][operations]\\
{\textbackslash}class{\{}d{\}}[class3][members]\\
{\textbackslash}class[rounded corners]{\{}e{\}}[other Object][members][operations]\\
{\textbackslash}rel{\{}generalization{\}}{\{}a{\}} {\{}b,c,d{\}}\\
{\textbackslash}rel{\{}dependency{\}}{\{}e{\}}{\{}d{\}}\\  
{\textbackslash}end{\{}diagram{\}}\\

\begin{diagram}{node distance=5cm,layered layout}
\class[rounded corners, fill=red!20]{a}[Interface\\Classes][members][operations]
\class[rounded corners, fill=blue!20]{b}[class1][members][operations]
\class{c}[class2][members][operations]
\class{d}[class3][members]
\class[rounded corners]{e}[other Object][members][operations]
\rel{generalization}{a} {b,c,d}
\rel{dependency}{e}{d}
    
\end{diagram}

\end{document}