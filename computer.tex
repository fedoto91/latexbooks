% LaTeX source for Introduction to Computer Science and Programming (Java)
% Primary Author:  Seth D. Bergmann

%% Diagram of a computer and peripherals


%% \documentclass{article}
%% \usepackage{../../cmds}
%% \usepackage{../../DraTex}
%% \usepackage{../../AlDraTex}
%% \usepackage{graphics}
%% \begin{document}
%% 
\begin {figure}

\Draw

\MinNodeSize(30,10)

\RectNode(cpu)(--CPU~~~~~~--)
\MoveToExit(2,0)
\RectNode(ram)(--RAM~~~~~~--)
\MoveToExit(4,3)
\OvalNode(diskTop)(-- --)
\MoveToExit(0,-2.3)

\MinNodeSize(30,30)
\Node(disk)(--Disk--)
\Move(-15,15)
\Line(0,-30)
\Move(30,0)
\Line(0,30)
\Move(-15,-30)
\DrawOvalArc(15,5)(180,0)

\Move(0,-50)
\MinNodeSize(40,20)
\RectNode(kb)(--KeyBoard~~~~--)
\SetBrush(5,5){\PaintRect(1,1)}
\Move(-20,-17)
\PaintRect(40,20)
\SetBrush(0,0){ }

\MoveToExit(0,-4.5)
\RRectNode(monitor)(--Monitor~~~~~~--)

% USB
%% \MoveToExit(-4.0,0)
%% \MarkLoc(usb1) \Move(10,0)
%% \MarkLoc(usb2) \Move(-5,10)
%% \MarkLoc(usb3) \Move(15,-20)
%% \PaintQuad(usb1,usb2,usb2,usb3)
%% 
%% \MarkLoc(sq)
%% \PaintRect(10,10)  \Move(-25,-5)
%% \PaintCircle(4)   \MarkLoc(circ) \Move(10, -35)
%% 
%% \PaintCircle(8)	  \MarkLoc(Circ)
%% 
%% \PenSize(1mm)
%% \Ragged(50)
%% 
%% \Line(0,50)   \MoveToLoc(circ) 
%% \MarkLoc(usb1) \Move(0,-30)
%% \MarkLoc(usb2) \MoveToLoc(Circ)  \Move(0,10)
%% \MarkLoc(usb4) \Move(-5,30)
%% \MarkLoc(usb3)
%% \Curve(usb1,usb2,usb3,usb4)
%% 
%% \MoveToLoc(Circ) 		\Move(0,15)
%% \MarkLoc(usb1)			\Move(5,30)
%% \MarkLoc(usb2)		\MoveToLoc(sq) 	\Move(0,-30)
%% \MarkLoc(usb3)
%% \Curve(usb1,usb2,usb3,sq)
%% 
%% \PenSize(0.75pt)
%% \Ragged(7.5)
%% 
%% xxx


\ArrowHeads(2)
\CurvedEdgeAt(ram,1,0.5,disk,-1,0)(0,0.1,180,0.1)

\ArrowHeads(1)
\CurvedEdgeAt(kb,-1,0,ram,1,0)(180,0.1,0,0.1)

\CurvedEdgeAt(ram,1,-0.5,monitor,-1,0)(0,0.1,180,0.1)



\EndDraw

%% \includegraphics*[5mm,5mm][20mm,30mm]{usb.eps}

\caption {Simplified diagram of a computer with peripheral 
devices}

\label {fig:computer}

\end {figure}

%%  \end {document}
